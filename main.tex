%%%%%%%%%%%%%%%%%%%%%%%%%%%%%%%%%%%%%%%%%
% Masters/Doctoral Thesis 
% LaTeX Template
% Version 1.43 (17/5/14)
%
% This template has been downloaded from:
% http://www.LaTeXTemplates.com
%
% Original authors:
% Steven Gunn 
% http://users.ecs.soton.ac.uk/srg/softwaretools/document/templates/
% and
% Sunil Patel
% http://www.sunilpatel.co.uk/thesis-template/
%
% License:
% CC BY-NC-SA 3.0 (http://creativecommons.org/licenses/by-nc-sa/3.0/)
%
% Note:
% Make sure to edit document variables in the Thesis.cls file
%
%%%%%%%%%%%%%%%%%%%%%%%%%%%%%%%%%%%%%%%%%

%----------------------------------------------------------------------------------------
%	PACKAGES AND OTHER DOCUMENT CONFIGURATIONS
%----------------------------------------------------------------------------------------

\documentclass[11pt, oneside]{Thesis} % The default font size and one-sided printing (no margin offsets)

\graphicspath{{Pictures/}} % Specifies the directory where pictures are stored

\usepackage[square, numbers, comma, sort&compress]{natbib} % Use the natbib reference package - read up on this to edit the reference style; if you want text (e.g. Smith et al., 2012) for the in-text references (instead of numbers), remove 'numbers' 
\hypersetup{urlcolor=blue, colorlinks=true} % Colors hyperlinks in blue - change to black if annoying
\title{\ttitle} % Defines the thesis title - don't touch this

\begin{document}

\frontmatter % Use roman page numbering style (i, ii, iii, iv...) for the pre-content pages

\setstretch{1.3} % Line spacing of 1.3

% Define the page headers using the FancyHdr package and set up for one-sided printing
\fancyhead{} % Clears all page headers and footers
\rhead{\thepage} % Sets the right side header to show the page number
\lhead{} % Clears the left side page header

\pagestyle{fancy} % Finally, use the "fancy" page style to implement the FancyHdr headers

\newcommand{\HRule}{\rule{\linewidth}{0.5mm}} % New command to make the lines in the title page

% PDF meta-data
\hypersetup{pdftitle={\ttitle}}
\hypersetup{pdfsubject=\subjectname}
\hypersetup{pdfauthor=\authornames}
\hypersetup{pdfkeywords=\keywordnames}

%----------------------------------------------------------------------------------------
%	TITLE PAGE
%----------------------------------------------------------------------------------------

\begin{titlepage}

\begin{center}
\textsc{\Large \bf \univname}\\[0.5cm] % University name
\textsc{\large Dipartimento di Ingegneria ``Enzo Ferrari"}\\[0.2cm] % Thesis type
%\HRule \\ % Horizontal line
\textsc{\large Corso di Laurea Magistrale in Ingegneria Informatica}\\[5.5cm] % Thesis type

\HRule \\[0.4cm] % Horizontal line
{\huge \bfseries \ttitle}\\[0.4cm] % Thesis title
\HRule \\[1.5cm] % Horizontal line
 
\begin{minipage}{0.4\textwidth}
\begin{flushleft} \large
\emph{Candidato:}\\
\authornames % Author name - remove the \href bracket to remove the link
\end{flushleft}
\end{minipage}
\begin{minipage}{0.4\textwidth}
\begin{flushright} \large
\emph{Relatore:} \\
\supname \\ % Supervisor name - remove the \href bracket to remove the link  
\emph{Correlatore:} \\
\cosupname \\ % Supervisor name - remove the \href bracket to remove the link  
\end{flushright}
\end{minipage}\\[7.5cm]
 
%\large \textit{A thesis submitted in fulfilment of the requirements\\ for the degree of \degreename}\\[0.3cm] % University requirement text
%\textit{in the}\\[0.4cm]
%\groupname\\\deptname\\[2cm] % Research group name and department name
 
\textsc{\large Anno Accademico 2013-2014}\\[4cm] % Date
%\includegraphics{Logo} % University/department logo - uncomment to place it
 
\vfill
\end{center}

\end{titlepage}

%----------------------------------------------------------------------------------------
%	DECLARATION PAGE
%	Your institution may give you a different text to place here
%----------------------------------------------------------------------------------------

%\Declaration{
%
%\addtocontents{toc}{\vspace{1em}} % Add a gap in the Contents, for aesthetics
%
%I, \authornames, declare that this thesis titled, '\ttitle' and the work presented in it are my own. I confirm that:
%
%\begin{itemize} 
%\item[\tiny{$\blacksquare$}] This work was done wholly or mainly while in candidature for a research degree at this University.
%\item[\tiny{$\blacksquare$}] Where any part of this thesis has previously been submitted for a degree or any other qualification at this University or any other institution, this has been clearly stated.
%\item[\tiny{$\blacksquare$}] Where I have consulted the published work of others, this is always clearly attributed.
%\item[\tiny{$\blacksquare$}] Where I have quoted from the work of others, the source is always given. With the exception of such quotations, this thesis is entirely my own work.
%\item[\tiny{$\blacksquare$}] I have acknowledged all main sources of help.
%\item[\tiny{$\blacksquare$}] Where the thesis is based on work done by myself jointly with others, I have made clear exactly what was done by others and what I have contributed myself.\\
%\end{itemize}
% 
%Signed:\\
%\rule[1em]{25em}{0.5pt} % This prints a line for the signature
% 
%Date:\\
%\rule[1em]{25em}{0.5pt} % This prints a line to write the date
%}

\clearpage % Start a new page

%----------------------------------------------------------------------------------------
%	QUOTATION PAGE
%----------------------------------------------------------------------------------------

%\pagestyle{empty} % No headers or footers for the following pages
%
%\null\vfill % Add some space to move the quote down the page a bit
%
%\textit{``Thanks to my solid academic training, today I can write hundreds of words on virtually any topic without possessing a shred of information, which is how I got a good job in journalism."}
%
%\begin{flushright}
%Dave Barry
%\end{flushright}
%
%\vfill\vfill\vfill\vfill\vfill\vfill\null % Add some space at the bottom to position the quote just right
%
%\clearpage % Start a new page

%----------------------------------------------------------------------------------------
%	ABSTRACT PAGE
%----------------------------------------------------------------------------------------

%\addtotoc{Abstract} % Add the "Abstract" page entry to the Contents
%
%\abstract{\addtocontents{toc}{\vspace{1em}} % Add a gap in the Contents, for aesthetics
%
%The Thesis Abstract is written here (and usually kept to just this page). The page is kept centered vertically so can expand into the blank space above the title too\ldots
%}
%
\clearpage % Start a new page

%----------------------------------------------------------------------------------------
%	DEDICATION
%----------------------------------------------------------------------------------------

\setstretch{1.3} % Return the line spacing back to 1.3

\pagestyle{empty} % Page style needs to be empty for this page

\dedicatory{To Emanuela} % Dedication text

\addtocontents{toc}{\vspace{2em}} % Add a gap in the Contents, for aesthetics

%----------------------------------------------------------------------------------------
%	ACKNOWLEDGEMENTS
%----------------------------------------------------------------------------------------

\setstretch{1.3} % Reset the line-spacing to 1.3 for body text (if it has changed)

\acknowledgements{\addtocontents{toc}{\vspace{1em}} % Add a gap in the Contents, for aesthetics

Foremost, I would like to express my sincere gratitude to my advisors Prof. Rita Cucchiara and Dr. Giuseppe Serra for the continuous support of my study, for their patience, motivation, enthusiasm, and knowledge. Their guidance helped me in all the time of research and writing of this thesis. I could not have imagined having better advisors for my study.

Besides my advisors, I would like to thank the rest of the Imagelab research group: Costantino Grana, Simone Calderara, Paolo Santinelli, Michele Fornaciari, Marco Manfredi, Franesco Solera, Martino Lombardi and Patrizia Varini, for their encouragement, insightful comments, and hard questions. I am also grateful to Prof. Greg Mori, for enlightening me the first glance of research during a visit in Modena.

Special thanks go to my labmate Francesco Paci, for the stimulating discussions, for the days we were working together before deadlines, and for all the fun we have had in the last months. Also I thank my friends Chiara Ferrari, Emanuele Benatti, Davide Setti, and in particular Michela Benedetti, who encouraged me to study ego-centric vision. Without her, this thesis would not probably exists.

Last but not the least, I would like to thank my parents Franco and Elisabetta, for giving birth to me at the first place and supporting me throughout my life, and my sister Alessia.
}
\clearpage % Start a new page

%----------------------------------------------------------------------------------------
%	LIST OF CONTENTS/FIGURES/TABLES PAGES
%----------------------------------------------------------------------------------------

\pagestyle{fancy} % The page style headers have been "empty" all this time, now use the "fancy" headers as defined before to bring them back

\lhead{\emph{Contents}} % Set the left side page header to "Contents"
\tableofcontents % Write out the Table of Contents

\lhead{\emph{List of Figures}} % Set the left side page header to "List of Figures"
\listoffigures % Write out the List of Figures

\lhead{\emph{List of Tables}} % Set the left side page header to "List of Tables"
\listoftables % Write out the List of Tables

%----------------------------------------------------------------------------------------
%	ABBREVIATIONS
%----------------------------------------------------------------------------------------

%\clearpage % Start a new page
%
%\setstretch{1.5} % Set the line spacing to 1.5, this makes the following tables easier to read
%
%\lhead{\emph{Abbreviations}} % Set the left side page header to "Abbreviations"
%\listofsymbols{ll} % Include a list of Abbreviations (a table of two columns)
%{
%\textbf{LAH} & \textbf{L}ist \textbf{A}bbreviations \textbf{H}ere \\
%%\textbf{Acronym} & \textbf{W}hat (it) \textbf{S}tands \textbf{F}or \\
%}

%----------------------------------------------------------------------------------------
%	PHYSICAL CONSTANTS/OTHER DEFINITIONS
%----------------------------------------------------------------------------------------

%\clearpage % Start a new page
%
%\lhead{\emph{Physical Constants}} % Set the left side page header to "Physical Constants"
%
%\listofconstants{lrcl} % Include a list of Physical Constants (a four column table)
%{
%Speed of Light & $c$ & $=$ & $2.997\ 924\ 58\times10^{8}\ \mbox{ms}^{-\mbox{s}}$ (exact)\\
%% Constant Name & Symbol & = & Constant Value (with units) \\
%}

%----------------------------------------------------------------------------------------
%	SYMBOLS
%----------------------------------------------------------------------------------------

%\clearpage % Start a new page
%
%\lhead{\emph{Symbols}} % Set the left side page header to "Symbols"
%
%\listofnomenclature{lll} % Include a list of Symbols (a three column table)
%{
%$a$ & distance & m \\
%$P$ & power & W (Js$^{-1}$) \\
%% Symbol & Name & Unit \\
%
%& & \\ % Gap to separate the Roman symbols from the Greek
%
%$\omega$ & angular frequency & rads$^{-1}$ \\
%% Symbol & Name & Unit \\
%}
\newcommand{\etal}{\textit{et al.~}}
%----------------------------------------------------------------------------------------
%	THESIS CONTENT - CHAPTERS
%----------------------------------------------------------------------------------------

\mainmatter % Begin numeric (1,2,3...) page numbering

\pagestyle{fancy} % Return the page headers back to the "fancy" style

% Include the chapters of the thesis as separate files from the Chapters folder
% Uncomment the lines as you write the chapters

% Chapter 1

\chapter{Egocentric Vision and Wearable devices: an overview}

\lhead{Chapter 1. \emph{Wearable devices and new human-machine interfaces}} % This is for the header on each page - perhaps a shortened title

%----------------------------------------------------------------------------------------

\section{Introduction}

Portable head-mounted cameras, able to record dynamic high quality first-person videos, have become a common
item among sportsmen over the last five years. These devices represent the first commercial attempts to record
experiences from a first-person perspective. This technological trend is a follow-up of the academic results
obtained in the late 1990s, combined with the growing interest of the people to record their daily activities.
As a recent survey on first person vision \cite{surveyfpv} recalls, the idea of recording and analyzing videos from first person perspective is not new. To mention some examples:
In 1998 Mann proposed the WearCam \cite{mann1998wearcam}. Later in 2000, Mayol \etal proposed a necklace device \cite{mayol2002wearable} , and in 2005 Mayol et al. developed an active shoulder mounted camera \cite{mayol2005applying}. In 2006, the Microsoft Research Center started to use the SenseCam for research purposes \cite{hodges2006sensecam}, while Pentland \etal \cite{blum2006insense} developed a wearable data collector system (InSense). Finally, it is important to
highlight the work of Mann, who, since 1978, has been working on his own family of devices. 

Up to date, no consensus has yet been reached in the literature with respect to naming this video perspective.
\textit{First Person Vision} (FPV) is probabily the most commonly used term, but also \textit{Egocentric Vision}
has also recently grown in popularity, and will be used in the rest of this thesis.  In the awakening of this technological trend, Google announced the Project Glass in 2012. The company started publishing short previews on the Internet demonstrating
the Glasses FPV recording capabilities. This was coupled by the ability of the device to show relevant information to the user through the head-up display. The main idea of the Project Glass is to use a wearable computer to reduce the time between intention and
action. In this thesis, we try to move a step forward, and realize a wearable vision device with better computing capacities and thus able to execute complex computer vision algorithms.

\section{On the positioning of the camera}
The ego-centric approach consists in equipping the users with wearable sensors that observe their activities: these devices see what the user sees and look where the user is looking. One of the first questions to solve, of course, is where to put the camera. Positioning an optical device on the human body is quite a problematic task, as occlusion, motion, social issues as well as criteria related to the purpose of the device must be taken into account. Following the work of Mayol \etal \cite{mayol2001positioning}, in this section we give a detailed overview on the best places where to put a wearable camera.

Cameras used for wearable applications fall into two categories for this discussion; static narrow-view devices and omnidirectional devices. Omnidirectional devices include catadioptric, fish-eye and active systems where either the entire field-of-viewis imaged at low resolution, or in the active case the high-resolution narrow-view sensor moved to any orientation. Narrow-view static cameras can only ever see a small part of the user or their environment, and placement is therefore entirely driven by the task. For wide-angle or omnidirectional sensors placement is less constrained and a range of positions are possible. 

A variety of solutions appear in the literature. In \cite{starner1998real, schiele1999attentional}, hat-mounted cameras have been used to look down at the user’s hands and reaching space, whereas in \cite{kohtake1999infostick} cameras are strapped to the wearer’s hands themselves. In \cite{aoki1999realtime}, a hat-mounted camera looks forward, an orientation also used when the camera is attached to a head mounted display \cite{mann1998wearcam}. In contrast, \cite{starner2000gesture} uses a camera worn on the chest, in \cite{rungsarityotin2000finding} an omnidirectional camera is used above the head, and a wide-angle lens camera mounted at the back in \cite{clarkson1999unsupervised}.

Mayol \etal \cite{mayol2001positioning} identify three frames of reference for measurements that a wearable sensor makes: 
\begin{enumerate}
\item relative to the user body (e.g. sensing the manipulative space in front of the user’s chest)
\item relative to the static world (e.g. sensing the ceiling/floor texture to infer user’s location)
\item relative to an independent object (e.g. tracking an interesting object)
\end{enumerate}
This task-oriented classification can help us to understand the criteria that should be considered. For working in the user frame alone all that is required is a stable view of the chosen area — often the handling space, and absolute field-of-view may be less important. For sensing the outside environment user occlusion is problematic and absolute field-of-view is more important. Both occlusion and
user motion are problems when fixating resolution or processing on a particular part of the environment or independently moving object.

To simulate and compare positions for optical devices around a human body, we must first simulate the human form: \cite{mayol2001positioning} use a female example from the Human Animation Working Group\footnote{\url{http://www.h-anim.org}}, consisting of about 1000 markers (points) and 1800 polygons arranged into 16 body-segments which can be independently rotated to simulate any natural pose. They have created a software\footnote{freely available at \url{http://www.robots.ox.ac.uk/~wmayol/3D/nancy_matlab.html}} to allow the simulated optical device to be positioned arbitrarily in space around the body, or for faster automatic tests placed a distance above any of the humanoid’s polygons. The utility of such a model is that the variables of position and distance above the body-surface can be varied automatically, allowing tests for a range of device heights over the whole body (which would be tedious at best on a real person).

Determination of occlusion in any direction can be made by emitting a ray from the chosen device centre and checking for intersection with any of the component polygons. Only polygons facing the camera need be considered, and refinements to further reduce the number of tests are widely reported in the ray-tracing literature. For visualization it is also useful to consider emitting rays from the device centre as equivalent to a central projection onto a unit sphere. This yields representations such as Figure \ref{fig:Positioning1}, where
the head is clearly visible to the right with the shoulder below it. The proportion of the sphere surface not occluded gives the absolute field-of-view.

\begin{figure}[htbp]
	\centering
		\includegraphics[page=3]{Figures/mayol_etal_ouel224101_cropped.pdf}
	\caption{Left: The device is positioned above the centroid of the polygon, a short distance (37.5mm) along the surface normal. Right: The view from such a camera shows the head and shoulder clearly, but the rest of the body is obscured.}
	\label{fig:Positioning1}
\end{figure}

In terms of absolute field of view, the head, the position of choice for many researchers (and nature), is favourable, as are the
shoulders. This comes as no surprise: however it is important to note which alternative placements are favorable if these locations can not be used (e.g. for social reasons). A further consideration is that raising the sensor away from the body might reduce the occlusion. Figure \ref{fig:Position2} shows the amount of occluision in some of the favoured positions and varies the sensor height from resting on the surface (height=0mm) to far above (height=150mm). At all of the chosen positions the occlusion reduces as the height is increased, but the most significant gains are made for the head-mounted positions and shoulder area. Whichever position is chosen, the field of view is always significantly improved if the sensor can be raised a few centimeters from the surface. However, increases beyond 50mm make no significant improvement in the field of view for this model.

\begin{figure}[htbp]
	\centering
		\includegraphics[page=4]{Figures/mayol_etal_ouel224101_cropped.pdf}
	\caption{The proportion of the view obscured against the distance from the body.}
	\label{fig:Position2}
\end{figure}

The amount of usermotion must also be taken into account. The type of motion that may be encountered can vary enormously and
may be task dependent. With an articulated model it is possible to analyze any motion for which the joint angle evolutions over time are known. In figure \ref{fig:Position3}, we show the amount of motion . The arms and legs suffer the highest
motions, with most of the torso and head relatively still.

\begin{figure}[htbp]
	\centering
		\includegraphics[page=8,width=0.4\linewidth]{Figures/mayol_etal_ouel224101_cropped.pdf}
	\caption{Amount of motion: arms and legs suffer the most severe motions during walking.}
	\label{fig:Position3}
\end{figure}

Now, having considered and evaluated occlusion, motion, and distance, it is clear that the best places to put a wearable camera, with respect to these parameters, are head and shoulders. Of course, an head mounted camera will suffer a bigger motion, but can reveal attentional cues, and it is thus the best choice for action and activity recognition. Shoulder mounted cameras, on the other hand, are less invasive, and are still a good choice for social interaction analysis and human-machine interfaces. In this thesis we will explore both options.

\section{Wearable devices}
The term \textit{wearable device} refers to electronic technologies or computer that are incorporated into items of clothing and accesories which can be worn on the body. In our case, we are mainly interested in developing wearable \textit{vision} devices, i.e. wearable devices capable to see and process video data. In this section, we will provide a brief description for some of these devices and our experience in using a few of them.

\subsection{Go-Pro}
GoPro offers a series of small and high-quality cameras. These cameras can be mounted on various
body parts or objects such as a helmet, chest and snow board which makes them very flexible. The
GoPro Hero 1 has a fish-eye lens with a 110 view angle. The view angle has increased to 170 in
the more recent Hero 2 and 3 cameras. The quality of the video is in HD, with minimal motion
blur and other artifacts. The quality of the video is much better outdoors, and the video sometimes
becomes very dark indoors. The main advantage of the GoPro camera is its wide field of view
and its high quality video. On the other hand, its disadvantage is its bulkiness, and the fact that
other individuals in the scene become very aware of it. The battery lasts for about 2 hours during
continuous video capture. The data is recorded on a SD card. A 16GB SD card suffices for storing
2 hours of HD video. The price of GoPro is around 200\$.

\subsection{Tobii Eye-Tracking Glasses}
Tobii offers various kinds of eye-tracking products. Most of their devices are static and monitorbased.
However, they have a few mobile eye-tracking systems as well. The system consists of an
outward looking camera that captures the scene in front of the user, and an inward looking infrared
camera that tracks the subject’s right eye. The glasses connect to a pocket size recording device.
Before or after data collection, the system needs to be calibrated in order to correctly estimate the
gaze point. Calibration is very intensive and becomes very hard for some subjects. The resolution
of the video is $640\times480$ and the frame rate is 30 fps. The video quality is low and there exist severe
motion blur and interlacing effects. On the plus size, the gaze tracking is accurate in comparison
to the wearable gaze-tracking devices of other companies. The view field of the camera is around
$600\times400$. The price of Tobii eye-tracking glasses is around 30,000\$.

\subsection{SMI Eye-Tracking Glasses}
SMI produced a device similar to Tobii’s eye-tracking glasses in January 2012. They have tried to fix some of the issues that exist in the Tobii’s system. In particular, their system is easier to calibrate, records a video in HD, and the glasses are more tolerable on the face. The SMI system has two eye-tracking infrared cameras looking at both eyes which results in an easier calibration in comparison to Tobii’s system. An issue that exists in the SMI glasses is that the video is blurred and dark on the frame boundaries. The price
of SMI eye-tracking glasses is around 24,000\$.

\subsection{Pivothead Glasses}
Pivothead has introduced a relatively cheap pair of glasses that have an outward looking camera
that captures the scene in front of the user. Obviously this is a cheaper system in comparison to
SMI and Tobii because it doesn’t track the eyes. The video quality is HD and it can capture for an
hour. The only issue with the pivothead glasses is that the camera’s field of view is very narrow,
even narrower than that of SMI and Tobii systems. The price of Pivothead glasses is around 300\$.
They also provide a 100\$ device which can transmit video to a laptop in realtime.

\subsection{Google Glass}
Google Glass is about to become available for public use. The system records a 720p video and
takes 5-megapixel images. It can connect to the internet and any bluetooth-capable phone. In
addition, Glass has a heads up display (HUD) creating an illusion equivalent to viewing a 25-inch
high definition screen from eight feet away. It has 16 GB of RAM, 12 GB of which are usable for
apps. Furthermore, it has a microphone, similar to all of the previously mentioned wearable devices.

\subsection{Panasonic HX-A10}
The Panasonic HX-A10, rather than relying on mounts and accessories, is equipped with an ''earhoo'' feature and a remote processing unit, which allows users to attach the camera to their head right out of the box, and no helmet is required. The device itself records at $1920\times1080$ video at 60 fps, $1280\times 720$ at 120 fps, and $640\times360$ at 240 fps for slow shots. It has Wi-Fi and can setup a live stream; and battery life for over two hours of filming on a single charge.

The actual lens (F2.5 bright lens, 1/4.1-inch BSI Sensor) part sports a two-foot cable and only weighs in at 4 ounces. It also communicates with a smartphone or tablet while watching in live view. We used the Panasonic HX-A10 to record two datasets, the Interactive Museum dataset and the Maramotti dataset, taken into the Maramotti Collection (Figure \ref{fig:Panasonic}). The lens helps to capture a wide field of view, which is necessary for observing first-person’s hands during daily activities. The fish-eye distortion, however, creates some challenges for using the data.


\begin{figure}[htbp]
	\centering
		\includegraphics[width=0.4\linewidth]{Figures/lore_maramotti.jpg}
	\caption{We used Panasonic HX-A10 for collecting the Interactive Museum dataset and the Maramotti dataset.}
	\label{fig:Panasonic}
\end{figure}

\subsection{Genius WideCam F100}
The Genius WideCam F100\footnote{\url{http://www.geniusnet.com/Genius/wSite/ct?xItem=53214&ctNode=161}} is a wide angle HD camera that we used in some of our tests, and is the camera mounted on our wearable vision device. It can record 1080p frames at up to 30 fps and features ultra wide angle lens (up to 120 degrees) and built-in microphones (see figure \ref{fig:wide-camera}).

\begin{figure}[t!]
\centering
\includegraphics[width=0.4\linewidth]{Figures/WideCam_F100_Angle.jpg}
\caption{Genius WideCam F100}
\label{fig:wide-camera}
\end{figure}

%----------------------------------------------------------------------------------------
\section{Previous work in Egocentric Vision}
We now discuss the relationship between this work with other areas of related work. Egocentric vision is an emerging area in computer vision. Every year around ten papers addressing this topic appear in top vision conferences (CVPR, ICCV and ECCV). It is possible to categorize these papers into roughly three main groups:
\begin{itemize}
\item Recognition of Activities
\item Gaze in Egocentric Vision
\item Day-long Video Summarization
\end{itemize}

In this chapter, we describe the related work to each of these topics.



%----------------------------------------------------------------------------------------
\section{About this thesis: gestures}
As we said before, the main aim of this thesis, from the hardware point of view, is to develop a wearable vision device capable of executing intensive computer vision algorithms. From the software point of view, we would like to build a new human-computer interface based on gestures.

Gestures are expressive, meaningful body motions involving
physical movements of the fingers, hands, arms, head, face, or
body with the intent of conveying meaningful information
or interacting with the environment. They constitute one interesting
small subspace of possible human motion. A gesture
may also be perceived by the environment as a compression
technique for the information to be transmitted elsewhere and
subsequently reconstructed by the receiver. Gesture recognition
has wide-ranging applications such as the following:
\begin{itemize}
\item developing aids for the hearing impaired;
\item  enabling very young children to interact with computers;
\item  designing techniques for forensic identification;
\item recognizing sign language;
\item medically monitoring patients’ emotional states or stress
levels;
\item lie detection;
\item navigating and/or manipulating in virtual environments;
\item communicating in video conferencing;
\item distance learning/tele-teaching assistance;
\item monitoring automobile drivers' alertness/drowsiness
levels, etc.
\end{itemize}

Generally, there exist many-to-one mappings from concepts
to gestures and vice versa. Hence, gestures are ambiguous and
incompletely specified. For example, to indicate the concept
\textit{stop}, one can use gestures such as a raised hand with palm
facing forward, or, an exaggerated waving of both hands over the
head. Similar to speech and handwriting, gestures vary between
individuals, and even for the same individual between different
instances.

Gestures can be static (the user assumes a certain pose or configuration)
or dynamic (with prestroke, stroke, and poststroke
phases). Some gestures also have both static and dynamic elements,
as in sign languages. Again, the automatic recognition
of natural continuous gestures requires their temporal segmentation.
Often one needs to specify the start and end points of a
gesture in terms of the frames of movement, both in time and
in space. Sometimes a gesture is also affected by the context of
preceding as well as following gestures. Moreover, gestures are
often language- and culture-specific. They can broadly be of the
following types:

%----------------------------------------------------------------------------------------

\section{In Closing}




\chapter{Hand segmentation in ego-centric videos}
As stated before, a hand segmentation mask is used to distinguish between camera and hand motions, and to prune away all the trajectories that do not belong to the user's hand. In this way, our descriptor captures hands movement and shape as if the camera was fixed, and disregards the noise coming from other moving regions that could be in the scene.

At each frame we extract superpixels using the SLIC algorithm \cite{achanta2012slic}, that performs a \textsl{k}-means-based local clustering of pixels in a 5-dimensional space, where color and pixel coordinates are used. Superpixels are then represented with several features: histograms in the HSV and LAB color spaces (that have been proven to be good features for skin representation \cite{khan2010skin}), Gabor filters and a simple histogram of gradients, to discriminate between objects with a similar color distribution.

\subsubsection{Illumination invariance}
To deal with different illumination conditions we train a collection of Random Forest classifiers indexed by a global HSV histogram, instead of using a single classifier. Hence, training images are distributed among the classifiers by a \textsl{k}-means clustering on the feature space. By using a histogram over all three channels of the HSV color space, each scene cluster encodes both the appearance of the scene and its illumination. Intuitively, this models the fact that hands viewed under similar global appearance will share a similar distribution in the feature space. Given a feature vector $\mathbf{l}$ of a superpixel $\mathbf{s}$ and a global appearance
feature $\mathbf{g}$, the posterior distribution of $\mathbf{s}$
is computed by marginalizing over different scenes $c$:

\begin{equation}
P(\mathbf{s}|\mathbf{l},\mathbf{g})=\sum_{c}P(\mathbf{s}|\mathbf{l},c)P(c|\mathbf{g})
\end{equation}


where $P(\mathbf{s}|\mathbf{l},c)$ is the output of a global appearance-specific
classifier and $P(c|\mathbf{g})$ is a conditional distribution of a
scene $c$ given a global appearance feature $\mathbf{g}$. In test
phase, the conditional $P(c|\mathbf{g})$ is approximated using an
uniform distribution over the five nearest models learned at training.
It is important to highlight that the optimal number of classifiers
depends on the characteristics of the dataset: a training dataset
with several different illumination conditions, taken both inside and
outside, will need an higher number of classifiers than one taken indoor.
In addition, we model the hand appearance not
only considering illumination variations, but also including semantic coherence in time and space.


\subsubsection{Temporal coherence}

To improve the foreground prediction of a pixel in a frame, we replace it with a weighted combination of its previous frames,
since past frames should affect the prediction for the current frame.

We define a smoothing filter for a pixel $x_{t}^{i}$ from frame $t$ as:
\begin{eqnarray}
P(x_{t}^{i} & = & 1)=\sum_{k=0}^{d}w_{k}(P(x_{t}^{i}=1|x_{t-k}^{i}=1)\cdot\nonumber \\
 &  & \cdot P(x_{t-k}^{i}=1|\mathbf{l}_{t-k},\mathbf{g}_{t-k})+P(x_{t}^{i}=1|x_{t-k}^{i}=0)\nonumber \\
 &  & \cdot P(x_{t-k}^{i}=0|\mathbf{l}_{t-k},\mathbf{g}_{t-k}))
\end{eqnarray}


where $d$ is the number of past frames used, and $P(x_{t-k}^{i}=1|\mathbf{l}_{t-k},\mathbf{g}_{t-k})$
is the probability that a pixel in frame $t-k$ is marked as hand
part, equal to $P(\mathbf{s}|\mathbf{l}_{t-k},\mathbf{g}_{t-k})$,
being $x_{t}^{i}$ part of $\mathbf{s}$. In the same way, $P(x_{t-k}^{i}=0|\mathbf{l}_{t-k},\mathbf{g}_{t-k})$
is defined as $1-P(\mathbf{s}|\mathbf{l}_{t-k},\mathbf{g}_{t-k})$.
Last, $P(x_{t}^{i}=1|x_{t-k}^{i}=1)$ and $P(x_{t}^{i}=1|x_{t-k}^{i}=0)$
are prior probabilities estimated from the training set as follows:

\[
P(x_{t}^{i}=1|x_{t-k}^{i}=1)=\frac{\#(x_{t}^{i}=1,x_{t-k}^{i}=1)}{\#(x_{t-k}^{i}=1)}
\]


\begin{equation}
P(x_{t}^{i}=1|x_{t-k}^{i}=0)=\frac{\#(x_{t}^{i}=1,x_{t-k}^{i}=0)}{\#(x_{t-k}^{i}=0)}
\end{equation}


where $\#(x_{t-k}^{i}=1)$ and $\#(x_{t-k}^{i}=0)$ are the number
of times in which $x_{t-k}^{i}$ belongs or not to a hand region,
respectively; $\#(x_{t}^{i}=1,x_{t-k}^{i}=1)$ is the number of times
that two pixels at the same location in frame $t$ and $t-k$ belong
to a hand part; similary $\#(x_{t}^{i}=1,x_{t-k}^{i}=0)$ is the number
of times that a pixel in frame $t$ belongs to a hand part and the
pixel in the same position in frame $t-k$ does not belong to a hand
region. 


\subsubsection{Spatial consistency}

Given pixels elaborated by the previous steps, we want to exploit
spatial consistency to prune away small and isolated pixel groups
that are unlikely to be part of hand regions and also aggregate bigger
connected pixel groups. For every pixel $x$, we extract its posterior
probability $P(x_{i}^{t})$ and use it as input for the GrabCut algorithm
\cite{rother2004grabcut}. Each pixel with $P(x_{i}^{t})\geq0.5$
is marked as foreground, otherwise it's considered as part of background.
After the segmentation step, we discard all the small isolated regions
that have an area of less than 5\% of the frame and we keep only the
three largest connected components.

\section{Experimental results}
\chapter{Towards ego-vision human-machine interfaces: gesture recognition}
\section{A distributed network of smart sensors to improve training}
\section{Support Vector Machines Hidden Markov Models}
\section{Experimental results}
\chapter{Conclusion}
\section{Publications}
%% Chapter 1

\chapter{Hand segmentation in ego-centric videos}

\lhead{Chapter 3. \emph{Hand segmentation}} % This is for the header on each page - perhaps a shortened title

%----------------------------------------------------------------------------------------
We now focus on the task of pixel-wise hand detection
from video recorded with a wearable head-mounted
camera. In contrast to a third-person point-of-view camera,
such as a mounted surveillance camera or a TV camera,
a first-person point-of-view wearable camera has exclusive
access to first-person activities and is an ideal viewing perspective
for analyzing fine motor skills such as hand-object
manipulation or hand-eye coordination. Recently, the use of
ego-centric video has re-emerged as a popular topic in computer
vision and has shown promising results in such areas
as understanding hand-eye coordination and recognizing
activities of daily living. In order to achieve more
detailed models of human interaction and object manipulation,
it is important to detect hand regions with pixel-level
accuracy. Hand detection is an important element of such
tasks as gesture recognition, hand tracking, grasp recognition,
action recognition and understanding hand-object interactions.

The egocentric
paradigm presents a new set of constraints and characteristics
that introduce new challenges as well as unique
properties that can be exploited for the task of first-person
hand detection. Unlike static third-person point-of-view
cameras typically used for gesture recognition or sign language
analysis, the video acquired by a first-person camera
undergoes large ego-motion because it is worn by the
user. The mobile nature of the camera also results in images
recorded over extreme transitions in lighting, such as
walking from indoors to outdoors. As a result, the large image displacement caused by body motion makes it very difficult
to apply traditional image stabilization or background
subtraction techniques. Similarly, large changes in illumination
conditions induce large fluctuations in the appearance
of hands. Fortunately, ego-centric videos also have the
property of being user-specific, where images of hands and
the physical world are always acquired with the same camera
for the same user. This implies that the intrinsic color
of the hands does not change drastically over time.

The purpose of this chapter is to identify and address the
challenges of hand detection for first-person vision. To this
end, we present a novel approach capable of facing
various illumination conditions and different backgrounds. Furthermore, we analyze our algorithm on existing datasets and propose a new challenging dataset. We perform extensive tests to highlight the pros and
cons of various widely-used local appearance features. We
evaluate the value of modeling global illumination to generate
an ensemble of hand region detectors conditioned on the
illumination conditions of the scene. Based on our finding,
we propose a model using sparse feature selection and an
illumination-dependent modeling strategy, and show that it
out-performs several baseline approaches.

\section{Literature overview}
The problem of hand detection  in ego-vision scenario, has been addressed only recently by the research community. Khan and Stoettinger in \cite{khan10} studied color classification for skin segmentation and pointed out how color-based skin detection has many advantages, like potentially high processing speed, invariance against rotation, partial occlusion and pose change. The authors tested Bayesian Networks, Multilayers Perceptrons, AdaBoost, Naive Bayes, RBF Networks and Random Forest. They demonstrated that Random Forest classification obtains the highest F-score among all the other techniques. 
Fathi \etal \cite{fathi11} proposed a different approach to hand detection, exploiting the basic assumption that background is static in the world coordinate frame. Thus foreground objects are detected as to be the moving region respect to the background. An initial panorama of the background is required to discriminate between background and foreground regions: this is achieved by fitting a fundamental matrix to dense optical flow vectors. 
This approach is shown to be a robust tool for skin detection for hand segmentation in a limited indoor environment but it performs poorly with more unconstrained scenes.

Li and Kitani \cite{li13} provide an historical overview about approaches for detecting hands from moving cameras. They define three categories: local appearance-based detection, global appearance-based detection, where a global template of hand is needed, and motion-based detection, which is based on the hypothesis that hands and background have different motion statistics. Motion-based detection approach requires no supervision nor training. On the other hand, this approach eventually identifies as hand an object manipulated by the user, since it moves together his hands. In addition they proposed a model with sparse feature selection which was shown to be an illumination-dependent strategy. To solve this issue, they trained a set of random forests indexed by a global color histogram, each one reflecting a different illumination condition.
Recently Bagdanov \etal \cite{bagdanov12} propose a method to predict the status of the user hand by jointly exploiting depth and RGB imagery.

All the presented previous works present good characteristics, but lack of generality, since they take into account only few aspects to model user hand appearance and they are not integrated with a gesture recognition system. We therefore present a novel method for hand segmentation and gesture recognition that can be used as basis for ego-vision applications. 
Hand detection is based on Random Forest classifiers learned by color and gradient features which are computed on superpixels. In order to improve the detection accuracy we present two strategies that incorporate temporal and spatial coherence: temporal smoothing and spatial consistency.

\section{Proposed approach}
Ego-vision applications require a fast and reliable segmentation of the hands; thus we propose to use random forest classifiers, as they are known to efficiently work even with large inputs \cite{leo01}. Since using a per-pixel basis in label assignment has show to be inefficient \cite{jones99}, we adopt segmentation method which assign labels to superpixels, as suggested in \cite{tighe13}. 
This allows a complexity reduction of the problem and also gives better spatial support for aggregating features that could belong to the same object. 

To extract superpixels for every frames we use the Simple Linear Iterative Clustering (SLIC) algorithm, proposed in \cite{achanta12} as memory efficient and highly accurate segmentation method. 
The SLIC super-pixel segmentation algorithm is a k-means-based local clustering of pixels in a 5D space, where Lab color values and pixel coordinates are used. A parallel implementation of the SLIC super-pixel algorithm is available in \cite{YHRengSLIC}. 

We represent superpixels by features to encode color and gradient information. As pointed out by previous works, HSV and LAB color spaces have been proven to be robust for skin detection. 
In particular, we describe each superpixel with mean and covariance matrix of its pixel values, and a 32-bin color histogram both in HSV and Lab color spaces.
To discriminate between objects with a similar color distribution of skin we include following gradient information: Gabor feature obtained with 27 filters (nine orientations and three different scales: $7 \times 7$, $13 \times 13$, $19 \times 19$) and a simple histogram of gradients with nine bins. 

\subsubsection{Illumination invariance}
To deal with different illumination conditions we train a collection of Random Forest classifiers indexed by a global HSV histogram, instead of using a single classifier. Hence, training images are distributed among the classifiers by a \textsl{k}-means clustering on the feature space. By using a histogram over all three channels of the HSV color space, each scene cluster encodes both the appearance of the scene and its illumination. Intuitively, this models the fact that hands viewed under similar global appearance will share a similar distribution in the feature space. Given a feature vector $\mathbf{l}$ of a superpixel $\mathbf{s}$ and a global appearance
feature $\mathbf{g}$, the posterior distribution of $\mathbf{s}$
is computed by marginalizing over different scenes $c$:

\begin{equation}
P(\mathbf{s}|\mathbf{l},\mathbf{g})=\sum_{c}P(\mathbf{s}|\mathbf{l},c)P(c|\mathbf{g})
\end{equation}


where $P(\mathbf{s}|\mathbf{l},c)$ is the output of a global appearance-specific
classifier and $P(c|\mathbf{g})$ is a conditional distribution of a
scene $c$ given a global appearance feature $\mathbf{g}$. In test
phase, the conditional $P(c|\mathbf{g})$ is approximated using an
uniform distribution over the five nearest models learned at training.
It is important to highlight that the optimal number of classifiers
depends on the characteristics of the dataset: a training dataset
with several different illumination conditions, taken both inside and
outside, will need an higher number of classifiers than one taken indoor.
In addition, we model the hand appearance not
only considering illumination variations, but also including semantic coherence in time and space.


\subsubsection{Temporal smoothing}
\begin{figure}[tb]
\centering
\includegraphics[width=.45\columnwidth]{./Figures/context-free2.jpg}
\includegraphics[width=.45\columnwidth]{./Figures/context-dependent2.jpg}
\caption{Comparison before (left image) and after (right image) Temporal smoothing.}
\label{fig:gesture_samples_time}
\end{figure}
We exploit temporal coherence to improve the foreground prediction of a pixel in a frame by a weighted combination of
its previous frames, since past frames should affect the results prediction for the current frame.

The smoothing filter for a pixel $\mathbf{x}_{t}^{i}$ of a frame $t$ (inspired by \cite{liu08}) can thus be defined as follows:

\begin{multline}
P(\mathbf{x}_{t}^{i}=1) = \sum_{k = 0}^{d} w_{k} \bigl( P(\mathbf{x}_{t}^{i}=1|\mathbf{x}_{t-k}^{i}=1) \cdot P(\mathbf{x}_{t-k}^{i}=1|\mathbf{l_{t-k}},\mathbf{g_{t-k}})\, + P(\mathbf{x}_{t}^{i}=1|\mathbf{x}_{t-k}^{i}=0) \, \cdot \\
\cdot P(\mathbf{x}_{t-k}^{i}=0|\mathbf{l_{t-k}},\mathbf{g_{t-k}}) \bigr)
\end{multline}
where $P(\mathbf{x}_{t-k}^{i}=1|\mathbf{l_{t-k}},\mathbf{g_{t-k}})$ is the posterior probability
that a pixel in frame $t-k$ is marked as hand part and $d$ is a number of past frames used. This likelihood can be defined as the probability $P(\mathbf{s}|\mathbf{l_{t-k}},\mathbf{g_{t-k}})$, being
$\mathbf{x}_{t}^{i}$ part of $\mathbf{s}$. In the same way, $P(\mathbf{x}_{t-k}^{i}=0|\mathbf{l_{t-k}},\mathbf{g_{t-k}})$ is defined as the probability
$1-P(\mathbf{s}|\mathbf{l},\mathbf{g_{t-k}})$. 

While $P(\mathbf{x}_{t}^{i}=1|\mathbf{x}_{t-k}^{i}=1)$ and $P(\mathbf{x}_{t}^{i}=1|\mathbf{x}_{t-k}^{i}=0)$ are prior probabilities estimated
from the training set as follows:

\[
P(\mathbf{x}_{t}^{i}=1|\mathbf{x}_{t-k}^{i}=1)=\frac{\#(\mathbf{x}_{t}^{i}=1,\mathbf{x}_{t-k}^{i}=1)}{\#(\mathbf{x}_{t-k}^{i}=1)}
\]


\[
P(\mathbf{x}_{t}^{i}=1|\mathbf{x}_{t-k}^{i}=0)=\frac{\#(\mathbf{x}_{t}^{i}=1,\mathbf{x}_{t-k}^{i}=0)}{\#(\mathbf{x}_{t-k}^{i}=0)}
\]


where $\#(\mathbf{x}_{t-k}^{i}=1)$ and $\#(\mathbf{x}_{t-k}^{i}=0)$ are the number of times in which $\mathbf{x}_{t-k}$ belongs or not to
a hand region, respectively; $\#(\mathbf{x}_{t}^{i}=1,\mathbf{x}_{t-k}^{i}=1)$ is the number
of times that two pixels at the same location at frame $t$ and $t-k$ belong to a hand part; 
similarly, $\#(\mathbf{x}_{t}^{i}=1,\mathbf{x}_{t-k}^{i}=0)$
is the number of times that a pixel in frame $t$ belongs to
a hand part and pixel in the same position in frame $t-k$ does not belong
to a hand region.
Figure \ref{fig:gesture_samples_time} shows an example where temporal smoothing deletes blinking regions (i.e. the tea box brand and jar shadows on the right).


\subsubsection{Spatial consistency}
\begin{figure}[tb]
\centering
\includegraphics[width=.45\columnwidth]{./Figures/context-free1.jpg}
\includegraphics[width=.45\columnwidth]{./Figures/context-dependent1.jpg}
\caption{Comparison before (left image) and after (right image) Spatial Consistency.}
\label{fig:gesture_samples_space}
\end{figure}
Given pixels elaborated by the previous steps, we want to exploit spatial consistency to prune away small and isolated pixel groups that are unlikely to be part of hand regions and also aggregate bigger connected pixel groups. 

For every pixel $\mathbf{x}$, we extract its posterior probability $P(\mathbf{x}_{t}^{i})$ and use it as input for the GrabCut algorithm \cite{rother04}. Each pixel with $P(\mathbf{x}_{t}^{i}) \geq 0.5$ is marked as foreground, otherwise it's considered as part of background. After the segmentation step, we discard all the small isolated regions that have an area of less than 5\% of the frame and we keep only the three largest connected components.

In Figure \ref{fig:gesture_samples_space} an example with and without applying the Spatial Consistency method is depicted; notice this technique allows to better aggregate superpixels that are near the principal blob region.


\begin{figure*}[tb]
\centering
\subfigure[]{\includegraphics[width=.30\columnwidth]{./Figures/index_sample.jpg}}
\subfigure[]{\includegraphics[width=.30\columnwidth]{./Figures/like_sample.jpg}}
\subfigure[]{\includegraphics[width=.30\columnwidth]{./Figures/dislike_sample.jpg}}
\subfigure[]{\includegraphics[width=.30\columnwidth]{./Figures/ok_sample.jpg}}
\subfigure[]{\includegraphics[width=.30\columnwidth]{./Figures/victory_sample.jpg}}
\caption{Our dataset consists of five gestures: a) point out; b) like; c) dislike; d) ok; e) victory.}
\label{fig:gesture_samples}
\end{figure*}

\section{Experimental results}
To evaluate the performance of proposed method we tested it on two datasets: EDSH and EGO-HSGR.
The recent publicly available EDSH dataset \cite{li13} consists in egocentric videos acquired to analyze performance of several hand detection methods. It consists in three videos (EDSH\_1, used as train video, and EDSH\_2 and EDSH\_{kitchen} used as test videos) that contain indoor and outdoor scenes with large variations of illumination, mild camera motion induced by walking and climbing stairs. All videos are recorded at a resolution of 720p and a speed of 30FPS. The dataset includes segmentation masks of hands, but it is not comprehensive of gesture annotations.     


In order to analyze the performance of our method to recognize gestures, we generated a new dataset which contains 12 videos of indoor scenes (EGO-HSGR); it includes segmentation masks and gesture annotations. Videos have been recorded with a Panasonic HX-A10 Wearable Camcorder at a resolution of $800 \times 450$ with a 25FPS in two different locations: a library and department's exhibition area.   

The aim of this dataset is to reproduce an environment similar to a museum for human and object interaction: paintings and posters are hung on the walls, true masterpieces or either its virtual images; the visitor walks and sometimes stops in front of an object of interest performing some gestures to interact with next generation wearable devices. We identify five different gestures that are used commonly: \textit{point out}, \textit{like}, \textit{dislike}, \textit{ok} and \textit{victory}. These can be associated to different action or used for record social experience. Fig. \ref{fig:gesture_samples} shows some frame examples. 

To evaluate performance of our pixel-level hand detector a subset of six videos are used (three for training and two for testing). Segmentation masks are provided every 25 frames for a total of 700 annotations. For gesture analysis we extract all the keyframes and we manually annotated them distinguishing between gestures.  
The F-score (harmonic mean of the precision and recall rate) is used to quantity hand detection performance, while gesture recognition is evaluated in terms of mAP (mean Average Precision).
%(we denote these videos as BB\_L1 and BB\_L2), a corridor of a faculty (BB\_F1 and BB\_F2) and a department (BB\_D1 and BB\_D2).
    
 \begin{table}
 \centering
 \begin{tabular}{|l|c|c|}
 \hline
 \textbf{Features} 	& \textbf{EDSH\_2} & \textbf{EDSH\_{kitchen}}	\\ \hline\hline
 HSV	& 0.752 & 0.801		\\ \hline
 + LAB	& 0.754	&	0.808 \\ \hline
 + LAB hist.	& 0.755 & 0.823			\\ \hline  
 + HSV hist. & 0.755 & 0.823			\\ \hline  
 + Grad hist. & 0.758	&	0.828	\\ \hline  
 + Gabor & 	\textbf{0.761}	& \textbf{0.829} \\ \hline  
 \end{tabular}
 \caption{Performance by incrementally adding new features.}\label{tab:localfeatures}
 \end{table}

\subsection{Features performance}

First, we examine the effectiveness of our features to discriminate between hand and non-hand superpixels. 
Table \ref{tab:localfeatures} shows performance in terms of F-measure on EDSH dataset with different feature combinations: firstly we describe each superpixel with mean and covariance matrix of its pixel values in HSV color space, then we do the same using LAB color space and we add color histograms. Lastly, we include a histogram of gradients and Gabor feature.
In order to analyze how visual features impact on the performance, in this experiment we do not include the temporal and spatial context information by using a single random forest classifier.     
Note that although color information plays a fundamental role for hand detection, some ambiguities between hands and other similar colored object still remain; these can be reduced by adding features based on gradient histograms. In fact, the usage of the full descriptor slightly improves the performance.      

\begin{table}
 \centering
 \begin{tabular}{|l|c|c|}
 \hline
 \textbf{Features} 	& \textbf{EDSH\_2} & \textbf{EDSH\_{kitchen}}	\\ \hline\hline
 II	& 0.789 & 	0.831		\\ \hline
 II + TS	& 0.791	&	0.834 \\ \hline
 II + TS + SC &	\textbf{0.852} &	\textbf{0.901}	\\ \hline  
 \end{tabular}
 \caption{Performance comparison considering Illumination Invariance (II), Time Smoothing (TS) and Spatial Consistency (SC).}\label{tab:context}
\end{table}


\subsection{Temporal Smoothing and Spatial Consistency}
In this experiment we validate the proposed techniques that take into account illumination variations, time dependence and spatial consistency.
Table \ref{tab:context} shows the F-measure scores obtained on EDSH dataset incrementally adding Illumination Invariance (II), Time Smoothing (TS) and Spatial Consistency (SC). 
Note that there is a significant improvement in performance when all these three techniques are applied together.   
In particular, illumination invariance substantially increases the performance with respect to results obtained using only visual features and a single random forest classifier, while the improvement introduced by temporal smoothing is less pronounced. The main contribution is given by Spatial Consistency, that prunes away small and isolated pixel groups and merge spatially nearby regions, increasing the F-measure score of about six percentage points.
The proposed technique is also tested in our EGO-HSGR dataset obtaining an F-measure score of $0.908$ and $0.865$ for the EGO-HSGR\_{4} and EGO-HSGR\_{5} videos.        


\begin{table}
 \centering
 \begin{tabular}{|l|c|c|}
 \hline
  	& \textbf{EDSH\_2}	& \textbf{EDSH\_{kitchen}} \\ \hline\hline
Hayman \etal \cite{hayman03} 	& 0.211 & 0.213		\\ \hline
Jones \etal \cite{jones99}	& 0.708 &	0.787	\\ \hline  
Li \etal \cite{li13} & 0.835 & 0.840		\\ \hline  
\textbf{Our method} & \textbf{0.852} &	\textbf{0.901}	\\ \hline 
\end{tabular}
\caption{Hand segmentation comparison with the state-of-the-art.}\label{tab:comparision_hand}
\end{table}



\subsection{Comparison to related methods}

In Table \ref{tab:comparision_hand} we compare our results to several approaches on EDSH dataset: a single-pixel color approach inspired by \cite{jones99}, a video stabilization approach based on background modeling using affine alignment of image frames inspired by
\cite{hayman03} and an approach based on random forest, proposed by \cite{li13}. 
The single-pixel approach is a random regressor trained only using single-pixel LAB color values. 
The background modeling approach aligns sequences of 15 frames estimating their mutual affine transformations; pixels with high variance are considered to be foreground hand regions. 
As can be seen, although the single-pixel approach is conceptually simple, is still quite effective. In addition, we observe that the low performance of the video stabilization approach is due to large ego-motion because the camera is worn by the user.     
The method recently proposed by \cite{li13} is more similar to our approach, but the use of superpixels, the selection of a new set of local features and the introduction of temporal and spatial consistency allow us to outperforms that results.
 
%% Chapter 1

\chapter{Gesture recognition in the cultural heritage scenario}

\lhead{Chapter 4. \emph{Hand gesture recognition}} % This is for the header on each page - perhaps a shortened title

%----------------------------------------------------------------------------------------
Having presented our hand segmentation approach, we move a step forward and introduce a more complex problem: detecting static and dynamic hand gestures. As we said in the introductory chapter, the target scenario of our research is the cultural heritage domain. In fact, our goal is to deploy new gesture-based human machine interfaces to enhance museum experience.

\section{Motivation}
In recent years the interest in cultural heritage has reborn, and the cultural market is becoming a cornerstone in many national economic strategies. In the United States, a recent report of the Office of Travel and Tourism Industries claims that 51\% of the 40 million Americans traveling abroad visit historical places; almost one third visit cultural heritage sites; and one quarter go to an art gallery or museum \cite{tourismintelligence}. The same interest is found in Europe, where the importance of the cultural sector is widely acknowledged, South Asia and North Africa. The latest annual research from World Travel and Tourism Council shows that travel and tourism's total contribution to total GDP grew by 3.0\% in 2013, faster than overall economic growth for the third consecutive year \cite{econotravel}.

Consequently, to deal with an increasing percentage of ``digital native'' tourists, a big effort is under way to propose new interfaces for interacting with the cultural heritage.
 In this direction goes the solution ``SmartMuseum'' proposed by Kuusik \etal \cite{kuusik2009smartmuseum}: by the means of PDAs and RFIDs, a visitor can gather information about what the museum displays, building a customized visit based on his or her interests inserted, prior to the visit, on their website. This project brought an interesting novelty when first released, but it has some limitations. First, being tied to RFIDs does not allow reconfiguring the museum without rethinking the entire structure of the exhibition. Furthermore, researches demonstrated how the use of mobile devices on the long term decreases the quality of the visit due to their users paying more attention to the tool rather than to the work of art itself.

In 2007 Kuflik \etal \cite{kuflik2011visitor} proposed a system to customize visitors experiences in museums using software capable of learning their interests based on the answers to a questionnaire that they compiled before the visit. Similarly to SmartMuseum, one of the main shortcomings of this system is the need to stop the visitor and force him into doing something that he/she might not be willing to do.
An interesting attempt to user profiling with wearable sensors was the Museum Wearable \cite{sparacino2002museum}, a wearable computer which orchestrates an audiovisual narration as a function of the visitors' interests gathered from his/her physical path in the museum. However this prototype does not use any computer vision algorithm for understanding the surrounding environment. For instance the estimation of the visitor location is based again on infrared sensors distributed in the museum space.

Museums and cultural sites still lack of an instrument that provides entertainment, instructions and visit customization in an effective natural way. Too often visitors struggle to find the description of the artwork they are looking at and when they finds it, its detail level could be too high or too low for their interests. Moreover, frequently the organization of the exhibition does not reflect the visitors' interests leading them to a pre-ordered path which cultural depth could not be appropriate.

To overcome these limitations, we try to enhance visitors' experiences using ego-vision. Ego-vision features glass-mounted wearable cameras able to see what the visitor sees and perceiving the surrounding environment as he does. Our goal is to develop a wearable vision device for museum environments, able to replace the traditional self-service guides and overcoming their limitations and allowing for a more interactive museum experience to all visitors. The aim of our device is to stimulate the visitors to interact with the artwork, reinforcing their real experience, by letting visitors to replicate the gestures (e.g. point out to the part of the painting they're interested in) and behaviors that they would use to ask a guide something about the artwork (see Fig. \ref{sample} for some samples of the recognized gestures).

In this chapter, we provide algorithms that perform gesture analysis to recognize user interaction with artworks. We also propose to use scalable and distributed wearable devices capable of communicating with each other and with a central server. In particular the connection with the central server allows our wearable devices to grab gestures of past visitors for improving gesture analysis accuracy.

The proposed gesture recognition algorithm is compared to the current state of the art on benchmark datasets showing superior performance, and is tested in real and virtual museum environments. We further demonstrate that our gesture recognition approach can achieve acceptable accuracy results even with a few training samples performed by the visitor, and can benefit from distributed training in which gestures performed by other visitors are exploited.


\section{Proposed Method}
To our knowledge, the study of gesture recognition in the ego-centric paradigm has not yet been addressed. Even though not related to ego-vision domain, several approaches to gesture and human action recognition have been proposed. Sanin \etal \cite{sanin2013spatio} developed a new and more effective spatio-temporal covariance descriptor to classify gestures in conjunction with a boost classifier. Lui \etal \cite{lui2010action, lui2011tangent} used tensors and tangent bundle on Grassmann manifolds to classify human actions and hand gestures. Kim \etal \cite{kim2009canonical} extended Canonical Correlation Analysis to measure video-to-video similarity to represent and detect actions in video. However, all these approaches are not appropriate for the ego-centric perspective, as they do not take into account any of the specific characteristics of this domain, such as fast camera motion and background cluttering.
\begin{figure}
\centering
\subfigure{
                \includegraphics[width=0.23\textwidth]{Figures/ok_7141_7164_text16.jpg}
                \includegraphics[width=0.23\textwidth]{Figures/dislike_3941_3969_text16.jpg}
} \\
\subfigure{
	    \includegraphics[width=0.23\textwidth]{Figures/slide_2_7438_7451_text5.jpg}
                \includegraphics[width=0.23\textwidth]{Figures/like_7038_7080_text23.jpg}
} \\
\subfigure{
                \includegraphics[width=0.23\textwidth]{Figures/point_15476_15505_text5.jpg}
                \includegraphics[width=0.23\textwidth]{Figures/take_a_picture_18622_18673_text37.jpg}
} \\
\caption{Results from the proposed hand segmentation and gesture recognition algorithms. Hand segmentation results are highlighted in red and detected gestures are reported in the bottom part of each frame.}
\label{sample}
\end{figure}

\begin{figure}
\centering
\includegraphics[width=\linewidth]{Figures/schema.pdf}
\caption{Outline of the proposed Gesture Recognition method.}
\label{schema}
\end{figure}
Gesture recognition systems should recognize both static and dynamic hand movements. Therefore, we propose to describe each gesture as a collection of dense trajectories extracted around hand regions. Feature points are sampled inside and around the user's hands and tracked during the gesture; then several descriptors are computed inside a spatio-temporal volume aligned with each trajectory, in order to capture its shape, appearance and movement at each frame. 
These descriptors are coded,  using the Bag of Words approach and power normalization, in order to obtain the final feature vectors, which are then classified using a linear SVM classifier. A summary of our approach is presented in Figure \ref{schema}.

In the next subsections we will introduce the main components of our approach: camera motion suppression, trajectory sampling and description, the power normalized Bag of Words and the final classification. We will also explain how these steps exploit the hand segmentation approach we have presented in the previous chapter.

\subsection{Feature points sampling}
We start by describing how we densely sample feature points for generating trajectories. To this aim, we consider a grid spaced by $W$ pixels. Sampling is carried out on each
spatial scale separately. This guarantees that feature points equally cover all spatial
positions and scales. Experimental results showed that a sampling step size of $W = 5$ pixels is dense
enough to give good results over all datasets. There are at most 8 spatial scales in total, depending on the
resolution of the video. The spatial scale increases by a factor of $1/\sqrt{2}$.
Our goal is to track all these sampled points through the video. However, in homogeneous image
areas without any structure, it is impossible to track any point. We remove points in these areas. Here, we
use the criterion that the points are removed if the eigenvalues of the auto-correlation
matrix are very small. We set a threshold $T$ on the eigenvalues for each frame $I$ as
\begin{equation}
T = 0.001 \times \max_{i\in I} \min(\lambda_i^1, \lambda_i^2)
\end{equation}

where $(\lambda_i^1, \lambda_i^2)$ are the eigenvalues of point $i$ in the image $I$. Experimental results showed that a value
of 0.001 represents a good compromise between saliency and density of the sampled points.

\subsection{Camera motion suppression}
A key step in our approach is to remove camera motion. To this end, the homography between two consecutive frames is estimated running the RANSAC algorithm on features points sampled as described. 

An homography (or \textit{projective transformation}) is defined by the following equation, where $x_1$ and $x_2$ are 2-d points expressed in homogeneous coordinates:
$$
x_1 = \left[\begin{array}{c c c}
 h_{11} & h_{12} & h_{13} \\
 h_{21} & h_{22} & h_{23} \\
 h_{31} & h_{32} & h_{33}
 \end{array}\right] x_2
$$
and, because two homography matrices that differ for a scale are equivalent, this matrix has only 8 degrees of freedom. RANSAC is accomplished to produce a good homography with the following steps:
\begin{enumerate}
\item Randomly selecting a subset of feature points
\item Fitting an homography to the selected subset
\item Determining the number of outliers
\item Repeating steps 1-3 for a prescribed number of iterations, and select the iteration with the minimum amount of outliers
\end{enumerate}

This would be the standard way to fit an homography between two frames. However, in first-person camera views hands movement is not consistent with camera motion and this generates wrong matches between the two frames. For this reason we introduce a segmentation mask that disregards feature matches belonging to hands. In fact, without the hand segmentation mask, many feature points from the user's hands would become inliers, degrading the homography estimation. As a consequence, the trajectories extracted from the video would be incorrect. Instead, computing an homography using feature points from non-hand regions allows us remove all the camera movements.

\subsection{Trajectory Sampling and description}
Having removed camera motion between two adjacent frames, trajectories can be extracted. The second frame is warped with the estimated homography, the optical flow between the first and the second frame is computed, and then feature points around the hands of the user are sampled and tracked following what \cite{wang:2011:inria-00583818:1} does for human action recognition.

Feature points are tracked on each spatial scale separately. For each frame $I_t$, its dense optical flow field
$\omega_t = (u_t, v_t)$ is computed w.r.t. the next frame $I_{t+1}$, where $u_t$ and $v_t$ are the horizontal and vertical
components of the optical flow. Given a point $P_t = (x_t, y_t)$ in frame $I_t$, its tracked position in frame $I_{t+1}$
is smoothed by applying a median filter on $\omega_t$:
\begin{equation}
P_{t+1} = (x_{t+1}, y_{t+1}) = (x_t, y_t) + (M * \omega_t)
\end{equation}
where $M$ is the median filtering kernel. The size of the median filter kernel $M$ is $3 \times 3$ pixels. As the
median filter is more robust to outliers than bilinear interpolation, it improves trajectories
for points at motion boundaries that would otherwise be smoothed out.
Once the dense optical flow field is computed, points can be tracked very densely without additional
cost. Another advantage of the dense optical flow is the smoothness constraints which allow relatively
robust tracking of fast and irregular motion patterns. To extract dense optical flow fields, we
use Farneback's algorithm \cite{farneback2003two} which embeds a translation motion model between neighborhoods of two consecutive
frames. Polynomial expansion is employed to approximate pixel intensities in the neighborhood.

Points of subsequent frames are concatenated to form trajectories: $(P_t,  P_{t+1}, P_{t+2}, ...))$. As trajectories
tend to drift from their initial locations during the tracking process, we limit their length to $L$ frames
in order to overcome this problem (based on our preliminary tests, we set $L=30$). For each frame, if no tracked point is found
in a $W \times W$ neighborhood, a new point is sampled and added to the tracking process so that a dense coverage
of trajectories is ensured.

As static trajectories do not contain motion information, we prune them in a post-processing stage.
Trajectories with sudden large displacements, most likely to be erroneous, are also removed. Such trajectories
are detected, if the displacement vector between two consecutive frames is larger than 70\% of the
overall displacement of the trajectory.

Furthermore, in contrast to what \cite{wang:2011:inria-00583818:1} does, trajectories are restricted to lie inside and around the user's hands: at each frame the hand mask is dilated, and all the feature points outside the computed mask are discarded.


Then, the spatio-temporal volume aligned with each trajectory is considered, and Trajectory descriptor, HOG, HOF and MBH are computed around it. The trajectory descriptors encodes local motion patterns. Given a trajectory of length $L$, we describe its
shape by a sequence $(\Delta P_t, ..., \Delta P_{t+L-1})$ of displacement vectors $\Delta P_t = (P_{t+1}- P_t) = (x_{t+1} -
xt, y_{t+1}- y_t)$. The resulting vector is normalized by the sum of displacement vector magnitudes:
\begin{equation}
T = \frac{(\Delta P_t, .., \Delta P_{t+L-1})}{\sum_{j=t}^{t+L-1} \| \Delta P_j \|}
\end{equation}
As we use trajectories with a fixed length of $L = 30$ frames, we obtain a 60 dimensional descriptor.

Besides the trajectory shape information, we also design descriptors to embed appearance and motion
information. We compute
descriptors within a space-time volume aligned with a trajectory to encode the motion information. The size of the volume is $N \times N$ pixels and $L$ frames long. To embed structure information, the volume is subdivided into a spatio-temporal grid of size $n_\sigma \times n_\sigma \times n_\tau$. We compute a descriptor (e.g.,
HOG, HOF or MBH) in each cell of the spatio-temporal grid, and the final descriptor is a concatenation
of these descriptors. The default parameters for our experiments are $N = 32$, $n_\sigma = 2$, $n_\tau = 3$.

As we have already said in Chapter \ref{chpt2}, HOG focuses on static appearance information, whereas HOF captures the
local motion information. We compute the HOG and HOF descriptors along the dense trajectories. For both HOG and HOF,
orientations are quantized into 8 bins with full orientation and magnitudes are used for weighting. An
additional zero bin is added for HOF (i.e., in total 9 bins). It accounts for pixels whose optical flow
magnitudes are lower than a threshold. Both descriptors are normalized with their $L_2$ norm. The final
descriptor size is 96 for HOG (i.e., $2\times 2\times 3\times 8$) and 108 for HOF (i.e., $2\times 2 \times 3 \times 9$).

The motion boundary histogram (MBH) descriptor computes derivatives separately for the horizontal and vertical components of the optical flow. The descriptor encodes the relative motion between pixels, and since MBH represents
the gradient of the optical flow, locally constant camera motion is removed and information about changes
in the flow field (i.e., motion boundaries) is kept. MBH is more robust to camera motion than optical flow,
and thus more discriminative.
We employ MBH s motion descriptor for trajectories. The MBH descriptor separates
optical flow $\omega = (u, v)$ into its horizontal and vertical components. Spatial derivatives are computed
for each of them and orientation information is quantized into histograms. The magnitude is used for
weighting. We obtain a 8-bin histogram for each component (i.e., MBHx and MBHy). Both histogram
vectors are normalized separately with their $L_2$ norm. The dimension is 96 (i.e., $2 \times 2 \times 3 \times 8$) for both
MBHx and MBHy.
For both HOF and MBH descriptor computation, we reuse the dense optical flow that is already
computed to extract dense trajectories.

While HOF and MBH are averaged on five consecutive frames, a single HOG descriptor is computed for each frame. In this way we can better describe how the hand pose changes in time. After this step, we get a variable number of trajectories for each gesture. In order to obtain a fixed size descriptor, the Bag of Words approach is exploited: we train four separate codebooks, one for each descriptor. Each codebook contains 500 visual words and is obtained running the $k$-means algorithm in the feature space. 

Since BoW histograms in our domain tend to be sparse, they are power normalized to unsparsify the representation, while still allowing for linear classification. To perform power-normalization \cite{perronnin2010improving}, the following function is applied to each bin $h_i$:
\begin{equation}
f(h_i) = \textrm{sign} (h_i) \cdot |h_i|^{\frac{1}{2}}
\end{equation}

We have also observed that power-normalization greatly improves the final accuracy. The feature vector is then obtained by the concatenation of its four power-normalized histograms. Eventually, gestures are recognized using a linear SVM 1-vs-1 classifier. SVM 1-vs-all and SVM Multiclass classifiers were also tested.

\section{Experimental Results}

We now evaluate the performance of the proposed approach. To compare the performance of our gesture recognition algorithm with existing approaches, we test it on the Cambridge-Gesture database \cite{kim2007tensor}, which includes nine hand gesture types performed on a table, under different illumination conditions. To better investigate the effectiveness of the proposed approach in videos taken from the ego-centric perspective and in a museum setting, we also propose two far more realistic and challenging dataset which contains seven gesture classes, performed by five subjects in an interactive exhibition room which functions as a virtual museum and in a real museum. 

The Cambridge Hand Gesture dataset contains 900 sequences of nine hand gesture classes. Although this dataset does not contain ego-vision videos it is useful to compare our results to recent gesture recognition techniques. In particular, each sequence is recorded with a fixed camera, placed over one hand, and hands perform leftward and rightward movements on a table, with different poses. The whole dataset is divided in five sets, each of them containing image sequences taken under different illumination conditions. The common test protocol, proposed in \cite{kim2007tensor}, requires to use the set with normal illumination for training and the remaining sets for testing, thus we use the sequences taken in normal illumination to generate the BoW codebooks and to train the SVM classifier. Then, we perform the test using the remaining sequences.   

Table \ref{cambridge} shows the recognition rates obtained with our gesture recognition approach, compared with the ones of tensor canonical correlation analysis (TCCA) \cite{kim2009canonical}, product manifolds (PM) \cite{lui2010action}, tangent bundles (TB) \cite{lui2011tangent} and spatio-temporal covariance descriptors (Cov3D) \cite{sanin2013spatio}. Results show that proposed method outperforms the existing state-of-the-art approaches.


\begin{table}
\begin{center}
\begin{tabular}{|l|c|c|c|c|c|}
\hline
\textbf{Method}					& \textbf{Set1}		& \textbf{Set2}		& \textbf{Set3}		& \textbf{Set4}	 	& \textbf{Overal}l \\
\hline
\hline
TCCA \cite{kim2009canonical}		& 0.81		& 0.81		& 0.78		& 0.86		& 0.82 \\
\hline
PM \cite{lui2010action}			& 0.89		& 0.86		& 0.89		& 0.87		& 0.88  \\
\hline
TB \cite{lui2011tangent}			& \textbf{0.93}	& 0.88		& 0.90		& 0.91		& 0.91 \\
\hline
Cov3D \cite{sanin2013spatio}		& 0.92		&\textbf{0.94}	& 0.94		& 0.93		& 0.93 \\
\hline
\textbf{Our method}			& 0.92		& 0.93		& \textbf{0.97}	& \textbf{0.95}	& \textbf{0.94} \\
\hline
\end{tabular}
\end{center}
\caption{Recognition rates on the Cambridge dataset.}
\label{cambridge}
\end{table}


\begin{figure}
\centering
\subfigure[\textit{Like} gesture in the \textit{museum} setting]{
  \includegraphics[width=0.2\linewidth]{Figures/like_16695_16718_all1.jpg}
\includegraphics[width=0.2\linewidth]{Figures/like_16695_16718_all14.jpg}
\includegraphics[width=0.2\linewidth]{Figures/like_16695_16718_all21.jpg}
}\\
\subfigure[\textit{Dislike} gesture]{
\includegraphics[width=0.2\linewidth]{Figures/dislike_10962_11000_all3.jpg}
\includegraphics[width=0.2\linewidth]{Figures/dislike_10962_11000_all23.jpg}
\includegraphics[width=0.2\linewidth]{Figures/dislike_10962_11000_all35.jpg}
}\\
\subfigure[\textit{Ok} gesture in the \textit{museum} setting, in low light]{
  \includegraphics[width=0.2\linewidth]{Figures/ok_2696_2735_all5.jpg}
\includegraphics[width=0.2\linewidth]{Figures/ok_2696_2735_all23.jpg}
\includegraphics[width=0.2\linewidth]{Figures/ok_2696_2735_all36.jpg}
}\\
\subfigure[\textit{Point} gesture]{
\includegraphics[width=0.2\linewidth]{Figures/point_4624_4654_all3.jpg}
\includegraphics[width=0.2\linewidth]{Figures/point_4624_4654_all8.jpg}
\includegraphics[width=0.2\linewidth]{Figures/point_4624_4654_all28.jpg}
}\\
\subfigure[\textit{Slide right to left} gesture in the \textit{museum} setting, while another visitor walks in]{
  \includegraphics[width=0.2\linewidth]{Figures/slide_2_7348_7383_all12.jpg}
\includegraphics[width=0.2\linewidth]{Figures/slide_2_7348_7383_all20.jpg}
\includegraphics[width=0.2\linewidth]{Figures/slide_2_7348_7383_all30.jpg}
}\\
\subfigure[\textit{Slide left to right} gesture in the \textit{demo room} setting performed by user 3]{
  \includegraphics[width=0.2\linewidth]{Figures/slide_2_2030_2058_all7.jpg}
\includegraphics[width=0.2\linewidth]{Figures/slide_2_2030_2058_all14.jpg}
\includegraphics[width=0.2\linewidth]{Figures/slide_2_2030_2058_all23.jpg}
}\\
\subfigure[\textit{Take a picture} gesture in the \textit{demo room} setting]{
  \includegraphics[width=0.2\linewidth]{Figures/take_a_picture_23956_24002_all2.jpg}
\includegraphics[width=0.2\linewidth]{Figures/take_a_picture_23956_24002_all25.jpg}
\includegraphics[width=0.2\linewidth]{Figures/take_a_picture_23956_24002_all43.jpg}
}
\caption{Sample gestures from the \datasetunimore{} dataset and the Maramotti dataset.}
\label{example-unimore}
\end{figure}

We then propose the \datasetunimore{} dataset, a gesture recognition dataset taken from the ego-centric perspective in a virtual museum environment. It consists of 700 video sequences, all shot with a wearable camera, in an interactive exhibition room, in which paintings and artworks are projected over a wall, in a virtual museum fashion (see figure \ref{example-unimore}). The camera is placed on the user's head and captures a 800 $\times$ 450, 25 frames per second 24-bit RGB image sequence. In this setting, five different users perform seven hand gestures: \textit{like}, \textit{dislike}, \textit{point}, \textit{ok}, \textit{slide left to right}, \textit{slide right to left} and \textit{take a picture}. Some of them (like the \textit{point}, \textit{ok}, \textit{like} and \textit{dislike} gestures) are statical, others (like the two \textit{slide} gestures) are dynamical. 
This dataset is very challenging since there is fast camera motion and users have not been trained before recording their gestures, so that each user performs the gestures in a slightly different way, as would happen in a realistic context. We have publicly released our dataset\footnote{\url{http://imagelab.ing.unimore.it/files/ego_virtualmuseum.zip}}.

%In the demo room setting, users perform gestures in front of a wall over which the works of art are projected. This setting is quite controlled: the illumination is constant, the art works are in low light, while hands are well illuminated. In the same way, in the \textit{museum} setting, users perform gestures in front of real artworks inside a museum. This is a realistic and very challenging environment: the illumination changes, other visitors are present and sometimes walk in. 
Since Ego Vision applications are highly interactive, their setup step must be fast (i.e. few positive examples can be acquired). Therefore, to evaluate the proposed gesture recognition approach, we train a 1-vs-1 linear classifier for each user using only two randomly chosen gestures per class as training set. The reported results are the average over 100 independent runs.

In Table \ref{gesture_comparison} we show the gesture recognition accuracy for each of the five subjects, and we also compare with the ones obtained without the use of the hand segmentation mask for camera motion removal and trajectories pruning.  Results show that our approach is well suited to recognize hand gestures in the ego-centric domain, even using only two positive samples per gesture, and that the use of the segmentation mask can improve recognition accuracy.


\begin{table}
\begin{center}
\renewcommand{\arraystretch}{1.3}
\centering
\begin{tabular}{|l|c|c|}
\hline
\textbf{User}							& \textbf{No segmentation} & \textbf{With segmentation}  \\
\hline
\hline
Subject 1				 			& 0.91 		& \textbf{0.95} \\
\hline
Subject 2			 				& 0.87 		& \textbf{0.87} \\
\hline
Subject 3					 		& 0.92		& \textbf{0.95}\\
\hline
Subject 4							& \textbf{0.96}	& 0.94 \\
\hline
Subject 5							& 0.91		& \textbf{0.96} \\
\hline
Average						&  0.91	& \textbf{0.93}  \\
\hline
\end{tabular}
\end{center}
\caption{Gesture recognition accuracy on the \datasetunimore{} dataset with and without hand segmentation.}
\label{gesture_comparison}
\end{table}

On a different note, to test our approach in a real setting, we created a dataset with videos taken in the Maramotti modern art museum, in which paintings, sculptures and \textit{objets d'art} are exposed. As in the previous dataset, the camera is placed on the user's head and captures a 800 $\times$ 450, 25 frames per second image sequence. The Maramotti dataset contains 700 video sequences, recorded by five different persons (some are the same of the Interactive Museum dataset), each performing the same gestures as before in front of different artworks. We have publicly released this dataset too\footnote{\url{ http://imagelab.ing.unimore.it/files/ego_maramotti.zip}}.
   
Figure \ref{example-unimore} show some examples of gestures performed in the two datasets. In the Interactive Museum dataset, users perform gestures in front of a wall over which the works of art are projected. This setting is quite controlled: the illumination is constant, the art works are in low light, while hands are well illuminated. On the other hand, in the Maramotti dataset, users perform gestures in front of real artworks inside a museum. This is a realistic and very challenging environment: the illumination changes, other visitors are present and sometimes walk in. In both cases there is significant camera motion, because the camera moves as the users move their heads or arms. It is also important to underline that users have not been trained before recording their gestures, so each user performs the gestures in a slightly different way, as would happen in a realistic context.  

In Table \ref{gesture_comparison_maramotti} we show the results of our gesture recognition approach on the Maramotti dataset. As can be seen, in this case the challenging and real environment causes a drop in accuracy. This is mainly due to the illumination changes, to the presence of other visitors, and to the fact that often the artworks are better illuminated than hands.
Since our wearable vision devices is fully connected to a central server, we show how the use of other visitors' gestures can improve the recognition accuracy.
In our scenario each visitor coming to the museum performs, in the initial setup phase, two training gestures for each class. These training gestures from past visitors, manually checked, are used to augment the training set, so no erroneous data is accumulated into the model. In particular, in our test ``Augmented'' (Table \ref{gesture_comparison_maramotti}) each ego-vision wearable device uses two randomly chosen gestures performed by its user as training, plus gestures performed by the remaining four users supplied by their devices to the central server. Results show that this distributed approach is effective and leads to a significant improvement in accuracy.


\begin{table}[tb]
\renewcommand{\arraystretch}{1.3}
\caption{Gesture recognition accuracy on the Maramotti dataset.}
\label{gesture_comparison_maramotti}
\centering
\begin{tabular}{|l|c|c|}
\hline
\textbf{User}					& \textbf{Single user's Gestures} 	& \textbf{Augmented}  \\
\hline
\hline
User A						& 0.54 			& \textbf{0.65} \\		% 4 Beppe
\hline
User B			 				& 0.52 			& \textbf{0.72} \\		% 2 Francesco
\hline
User C			 			& 0.68 			& \textbf{0.68} \\		% 1 Lorenzo
\hline
User F					 		& 0.56 			& \textbf{0.79} \\		% 3 Stefano
\hline
User G						& 0.53 			& \textbf{0.72} \\		% 5 Michela
\hline
\textbf{Average}					& 0.57	  		& \textbf{0.71} \\
\hline
\end{tabular}
\end{table}

We described a novel approach to cultural heritage fruition based on ego-centric vision devices. Our work is motivated by the increasing interest in ego-centric vision and by the growth of the cultural market, which encourages the development of new interfaces to interact with the cultural heritage. We presented a gesture and painting recognition model that can deal with static and dynamic gestures and can benefit from a distributed training. Our gesture recognition and hand segmentation results outperform the state-of-the-art approaches on Cambridge Hand Gesture and CMU EDSH datasets. Finally, we ran an extensive performance analysis of our system on a wearable board. 

In the next chapter, a real-time version of the described algorithm will be proposed, and we will describe the necessary optimizations steps to make such implementation real-time.
%% Chapter 1

\chapter{A real-time implementation for the Odroid-XU developer board}

\lhead{Chapter 5. \emph{A real-time implementation}} % This is for the header on each page - perhaps a shortened title

%----------------------------------------------------------------------------------------
Another important contribution of this thesis is a real-time implementation of our gesture recognition approach, ready to be used on an embedded device and suitable for real-world applications. To achieve this goal, we will need to modify and improve the classification module of our algorithm, and choose the Odroid-XU developer board as our target plaform. To make our implementation capable of running in real-time, we will also have to exploit several optimization techniques. At the end of this chapter we will also present some real-world applications of the proposed algorithm.

In the previous chapter, we classified frame sequences containing a gesture using a linear and power-normalized SVM classifier. In particular, we extracted trajectories with fixed length $L=30$ from the frame sequence, we then applied a standard BoW approach, and the output of the BoW was our final feature vector, with fixed size and ready to be classified. We now ask ourselves how to modify this approach in order to take a frame stream (with unknown length) as input, instead of a fixed sized sequence. To this aim, we propose two approaches: a naive sliding window approach and a more complex label sequence learning approach. In the following, we will use the latter as our default implementation.

\section{How to treat a frame stream}
The simplest way to extend the proposed approach to frame streams is to exploit a sliding window. Let's suppose we have a window with size $W$, where $W$ is greater than the maximum gesture duration, and that the window step is $s$. Thus, at iteration $i$, the window will contain the following set of frames:
\begin{equation}
\{ I_{1+i\cdot s}, I_{2+i \cdot s}, ..., I_{W+i \cdot s} \}
\end{equation}

each window is then classified using the linear SVM classifier. Of course, in this case we have an additional class, the \textit{non-gesture} one. We define as \textit{non-gesture} any window that does not contain a complete gesture.

However, this implementation leads to, at least, three drawbacks: first, we would be always $s$ frames late, and therefore for a real-time gesture recognition we would have to choose a small $s$ and accept to classify the same frame lots of times. Secondly, the classifier wouldn't learn a clear concept of \textit{non-gesture}, since the \textit{non-gesture} class would contain partial gestures. Third, an SVM classifier doesn't learn label/label dependencies.

\medskip

For this reasons, we will now treat the problem of classifying a frame stream as a \textit{label sequence learning} problem. We start observing that our trajectory descriptors can be thought as a concatenation of single-point descriptors. Formally, let's consider a trajectory $T_i$ starting at frame $t_i$ and ending at frame $t_i+L-1$: we can express $T_i$ as $\left[ P_{t_i}^i, P_{t_{i+1}}^i, ..., P_{t_i+L-1}^i \right]$, where $P_{j}^i$ is the point of trajectory $T_i$ at frame $j$.  If we denote with $D(T_i)$ the descriptor of trajectory $T_i$, then $D(T_i)$ can be expressed as the concatenation of $d(P_{t_i}^i), d(P_{t_{i+1}}^i), ..., d( P_{t_i+L-1}^i)$, where $d(P_{j}^i)$ is a descriptor computed around $P_j^i$, at frame $j$. Proving this statement is straightforward, given the nature of the descriptors employed.

Let's now consider a single frame $j$ of a frame stream, and suppose $ \{ T_1, T_2, ..., T_n\}$ are the trajectories crossing this frame. In this on-line version, we build a BoW histogram on each frame, being the input of the BoW clustering the set of descriptors of the trajectory points crossing the frame, i.e. $\{ d(P_{j}^1),  d(P_{j}^2), ...,  d(P_{j}^n) \}$.  Therefore, the corresponding BoW histogram won't represent a collection of trajectories anymore, but a collection of points, each one belonging to a different trajectory, and all related to the same frame.

Shortly, in the previous chapter the temporal dimension was encoded directly into the BoW input, now this will be up to the sequence classifier.


\subsection{Classification}
 In Section \ref{svmhmm} we described how a Struct SVM classifier can be used for label sequence learning. Now, we exploit the Structural SVM implementation by Altun \etal \cite{altun2003hidden} to classify this modified version of our feature vectors.

 First of all, we modify the source code provided by Joachims \cite{joachims} in order to include higher order label-label and label-features interactions, so that the generic 
$\Psi(\mathbf{x},\mathbf{y})$ takes the following form:

\begin{equation}
\Psi_{\epsilon,\tau}(\mathbf{x},\mathbf{y}) = \left( \begin{array}{cc} \sum_{t=1}^T \Phi(\mathbf{x}^t) \otimes  \Lambda^c(y^{t-\epsilon}) \otimes ... \otimes \Lambda^c(y^t) \\ \eta\sum_{t=1}^{T}  \Lambda^c(y^{t-\tau}) \otimes ... \otimes \Lambda^c(y^t) \end{array} \right)
\end{equation}

where $\epsilon$ is the order of label-label dependencies, and $\tau$ is the order of label-features dependencies.

Furthermore, we investigate the use of different loss functions and their impact on recognition rates and confusion matrices. We define a generic loss function that exploits a per-token loss function $\delta(y^t,y'^t)$: $\Delta(\mathbf{y},\mathbf{y'}) = \sum_{t=1}^T \delta(y^t,y'^t)$, and propose two different per-token loss functions:

\begin{equation}
\delta_0(y,y') = \begin{cases} 0 \quad \text{if } y= y'  \\ 
1 \quad \text{otherwise}\end{cases}
\label{eq:loss-default}
\end{equation}

\begin{equation}
\delta_1(y,y') = \begin{cases} 0 \quad \text{if } y= y'  \\ 
3 \quad \text{if } y=y_{ng}, y' \neq y_{ng} \lor y\neq y_{ng}, y'=y_{ng}  \\
1 \quad \text{otherwise}\end{cases}
\label{eq:loss-mia}
\end{equation}

where $y_{ng}$ is the \textit{non-gesture} label. \ref{eq:loss-default} is the default 0-1 loss function included in Joachims's code, and \ref{eq:loss-mia} is a slightly modifed version that aims at penalizing the case when a non-gesture frame is confused with a gesture frame or viceversa.

\begin{table}
\begin{center}
\begin{tabular}{|c|c|c|}
\hline
$\mathbf{\tau}$ & $\mathbf{\epsilon}$ & \textbf{Accuracy \%} \\
\hline
\hline
1							& 0							& 92.2 \% \\
1							& 1							& 92.4 \% \\
2							& 0							& 92.4 \% \\
2							& 1							& 93.0 \% \\
\textbf{2}						& \textbf{2}						& \textbf{93.8 \%} \\
\hline
\end{tabular}
\end{center}
\caption{Accuracy results with different orders of dependencies of transitions and emissions}
\label{transemis}
\end{table}

\begin{table*}
\begin{center}
\begin{tabular}{|c|c|c|c|c|c|c|c|c|}
\hline
 	& \textbf{Like}	& \textbf{Dislike}	&\textbf{Point}	&\textbf{Ok}	&\textbf{Slide LR}	&\textbf{Slide RL}	&\textbf{T. a pic}	& \textbf{No ges.} \\
\hline
\hline
 \textbf{Like}   & 80\% & 0 \% & 0 \% & 10\% & 0 \% & 0 \% & 0 \% & 10\% \\
\textbf{Dislike}   & 0 \% & 91\% & 0 \% & 0 \% & 0 \% & 0 \% & 0 \% &  9\% \\
\textbf{Point}  & 0 \% & 0 \% & 91\% & 9\% & 0 \% & 0 \% & 0 \% & 0 \% \\
\textbf{Ok}  & 0 \% & 0 \% & 0 \% & 100\% & 0 \% & 0 \% & 0 \% & 0 \% \\
\textbf{Slide left to right}  & 0 \% & 0 \% & 0 \% & 0 \% & 90\% & 10\% & 0 \% & 0 \% \\
\textbf{Slide right to left}  & 0 \% & 0 \% & 0 \% & 0 \% & 0 \% & 100\% & 0 \% & 0 \% \\
\textbf{Take a picture}  & 0 \% & 0 \% & 0 \% & 0 \% & 0 \% & 0 \% & 100\% & 0 \% \\
\textbf{No gesture}   & 1\% & 0 \% & 0 \% & 1\% & 1\% & 0 \% & 3\% & 93\% \\
\hline
\end{tabular}
\end{center}
\caption{Confusion matrix using the $\delta_0$ per-token loss function. Percentages are rounded to the nearest integer and computed considering sequences of adjacent frames in which a gesture is performed or not.}
\label{losses0}
\end{table*}

\begin{table*}
\begin{center}
\begin{tabular}{|c|c|c|c|c|c|c|c|c|}
\hline
 	& \textbf{Like}	& \textbf{Dislike}	&\textbf{Point}	&\textbf{Ok}	&\textbf{Slide LR}	&\textbf{Slide RL}	&\textbf{T. a pic}	& \textbf{No ges.} \\
\hline
\hline
 \textbf{Like}  & 90\% & 0 \% & 10\% & 0 \% & 0 \% & 0 \% & 0 \% & 0 \% \\
\textbf{Dislike}  & 9\% & 91\% & 0 \% & 0 \% & 0 \% & 0 \% & 0 \% & 0 \% \\
\textbf{Point}  & 0 \% & 0 \% & 91\% & 9\% & 0 \% & 0 \% & 0 \% & 0 \% \\
\textbf{Ok}  & 0 \% & 0 \% & 0 \% & 100\% & 0 \% & 0 \% & 0 \% & 0 \% \\
\textbf{Slide left to right}  & 0 \% & 0 \% & 0 \% & 0 \% & 90\% & 10\% & 0 \% & 0 \% \\
\textbf{Slide right to left} & 0 \% & 0 \% & 0 \% & 0 \% & 0 \% & 100\% & 0 \% & 0 \% \\
\textbf{Take a picture} & 0 \% & 0 \% & 0 \% & 0 \% & 0 \% & 0 \% & 100\% & 0 \% \\
\textbf{No gesture}  & 0 \% & 0 \% & 0 \% & 1 \% & 0 \% & 0 \% & 1 \% & 97\% \\
\hline
\end{tabular}
\end{center}
\caption{Confusion matrix using the $\delta_1$ per-token loss function.  Percentages are rounded to the nearest integer and computed considering sequences of adjacent frames in which a gesture is performed or not.}
\label{losses1}
\end{table*}

We train the Structural SVM classifier using four different gesture sequences from the Maramotti dataset, each five minutes long, and perform test on three sequences taken in front of different artworks. Using our gesture recognition algorithm we get a feature vector for each frame, which is then classified using the Structural SVM classifier. Of course, since sequences contain frames in which the user is not performing any gesture, the classifier has to deal with a \textit{non-gesture} class too.

Table \ref{transemis} shows the accuracy results obtained with various orders of label-label and label-features dependencies, using the $\delta_0$ per-token loss function. As can be seen, high accuracy results can be achieved even with $\tau = 1$ and $\epsilon = 0$, whereas using second order dependencies there is a slight increase in accuracy. On the other hand, this would imply slower training and testing.

Moreover, the $\delta_0$ per-token loss function leads to a confusion matrix where some \textit{non-gesture} examples are classified as gestures (see Table \ref{losses0}). Since false positives can be a major problem in human-machine interfaces, we propose the $\delta_1$ loss function, which increases the penalty when \textit{non-gesture} examples are misclassified and viceversa. Table \ref{losses1} shows that this loss function significantly reduce the confusion between gesture and \textit{non-gesture} sequences.


\section{Implementation}
\lstset{language=C++}

As stated in Section \ref{odroid-intro}, on the Odroid-XU board only one of the two clusters can run at the same time. Since our application is CPU intensive, we choose the A15 cluster, which includes four cores and is the most performant cluster. This is done trough the following command:
\begin{lstlisting}[frame=single]  % Start your code-block

echo performance > /sys/devices/system/cpu/cpu0/cpufreq/scaling_governor
\end{lstlisting}
which basically activates the \verb+performance+ governor as the controller for the big.LITTLE switching mechanism. The \verb+performance+ governor will simply turn off the A7 cluster and turn on the A15 cluster.

Then, having written a sequential implementation of our algorithm using the OpenCV library and our modified version of Joachims's SVM-HMM \cite{joachims} \footnote{This is joint work with Francesco Paci.}, we compile it using \verb+gcc+ and the following options:
\begin{lstlisting}[frame=single]  % Start your code-block

-g -mfpu=neon-vfpv4 -ftree-vectorize -mfloat-abi=hard -mtune=cortex-a15 -marm
\end{lstlisting}

The \verb+-g+ flag tells the compiler to produce debugging information in the operating system's native format, \verb+-mfpu=neon+  specifies what floating-point hardware (or hardware emulation) is available on the target, \verb+-ftree-vectorize+ enables the auto-vectorizer (which will try to use SIMD instructions, when possible), \verb+-mfloat-abi=hard+ allows generation of floating-point instructions and uses FPU-specific calling conventions, and \verb+-mtune=cortex-a15+ tunes the performance of the compiler for the A15 target.

\subsection{Optimizations}
\begin{figure}[t!]
\centering
\includegraphics[width=1.5\linewidth,angle=90]{Figures/grafico_prestazioni.jpg}
\caption{A section of the Valgrind Call Graph of the first sequential version. As can be seen, the optical flow takes more than the 50\% of the execution time.}
\label{valgrind}
\end{figure}

This first sequential version runs at 4.3 frames/s on $160\times 120$ frames, and therefore needs 234 ms to elaborate each frame. Having profiled the application with Valgrind \cite{nethercote2007valgrind}, we observe that the main bottleneck of this implementation is the multi-scale Farneback's optical flow, which requires 141 ms for each frame to run (see Figure \ref{valgrind} for a section of the Valgrind call graph). Farneback's algorithm is executed two times for each frame, since the optical flow is calculated both on the original frame on the warped one, so we exploit OpenMP parallel sections to run these two calls simultaneously on two different threads. Then, we use Neon intrinsics in order to exploit the SIMD capabilities of our CPU and thus gain a better performance on each thread.

A complete explanation of the optimizations made is beyond the purpose of this chapter, but, as an example, let's consider this for loop, inside the OpenCV implementation of Farneback's optical flow\footnote{Source code is available at: \url{https://github.com/Itseez/opencv/blob/0224a20ff6d0cf051cf818efb364048a2dcb716d/modules/video/src/optflowgf.cpp}}, which is responsible of 51 of the 234 ms:
\begin{lstlisting}[frame=single]  % Start your code-block

for( ; x < width*5; x++ ) {
	float s0 = srow[m][x]*kernel[0];
	for( i = 1; i <= m; i++ )
		s0 += (srow[m+i][x] + srow[m-i][x])*kernel[i];
	vsum[x] = s0;
}
\end{lstlisting}

Once optimized with Neon SIMD, the previous code becomes:
\begin{lstlisting}[frame=single]  % Start your code-block

int xstart = x;
float kernelext0[4];
fill_n(kernelext0,4,kernel[0]);
for( ; x < width*5-4; x+=4 ) {
           vst1q_f32(vsum+x, vmulq_f32(vld1q_f32(srow[m]+x),
		vld1q_f32(kernelext0)));
}
float a[width*5];
float kernelext[4];
for( i = 1; i <= m; i++ ) {
	fill_n(kernelext, 4, kernel[i]);
           for (x=xstart; x<width*5-4; x+=4) {
                        vst1q_f32(a+x, vaddq_f32(vld1q_f32(srow[m+i]+x), 
			vld1q_f32(srow[m-i]+x)));
                        vst1q_f32(vsum+x, vmlaq_f32(vld1q_f32(vsum+x), 
			vld1q_f32(a+x), vld1q_f32(kernelext)));
           }
}
\end{lstlisting}
As can be seen, the two for loops have been inverted and four floats are processed at each iteration now. The new code block takes only 20 ms to run.


Having included several other OpenMP/SIMD optimizations, our code runs at 10 frames/s on the A15 cluster, still not enough for real-time. Moreover, our algorithm needs an high frame rate in order to compute significant trajectories. Being Farneback's algorithm our main bottleneck, we could turn our attention to the PowerVR GPU, and use the OpenCL implementation of Farneback's algorithm included in OpenCV, for instance. Unfortunately, Hardkernel has not yet released an Ubuntu kernel that supports the PowerVR GPU\footnote{See: \url{http://forum.odroid.com/viewtopic.php?f=61&t=2236}.}, so the only way we have to increase speed is to further reduce the frame size and keep only one level of the spatial pyramid. Having reduced the frame size to $113\times 85$ the code can run at 14 fps, which is quite a good result, since trajectories can still be extracted with good accuracy and the overall recognition performance is only slightly affected: in fact, we have observed a $5\%$ drop in recognition accuracy.

\section{Some applications}
During the process of adapting our approach to build a real-time gesture recognizer we have implemented and tested two applications, both of which have required a careful tuning and test phase: an \textit{ego-vision jacket}, which basically is a jacket that embeds our developer board and a camera, placed on the chest, and a \textit{gesture-based interface} for desktop applications.

\textbf{Ego-Vision Jacket}: Embedding a wearable camera and a board in a jacket has required some tailoring work: the lens of the camera has been placed on a butthole, sewed on the chest, and the board in a custom designed pocket, made with breathable fabric. Furthermore, a battery has been developed to make the board completely wireless. 

\begin{figure}[t!]
\centering
\includegraphics[width=1.2\linewidth,angle=90]{Figures/consumi.pdf}
\caption{CPU Frequency, Power consumption and CPU Temperature during a two hours execution of our algorithm, inside the pocket of Ego-vision Jacket.}
\label{jacket}
\end{figure}

Of course here the main technical issue is overheating, since the board has to stay inside a pocket: for this reason, we extensively measured the CPU temperature with different workload conditions and external temperatures. Results during a two hours execution of our algorithm shows that in fact the board temperature is remarkably higher than in normal conditions, but the embedded fan is still able to maintain the cores well below their maximum allowed temperature, that is $80\text{ }^{\circ}\text{C}$ (see Fig. \ref{jacket}).

Our \textit{ego-vision jacket} is the first prototype of a future high technological jacket we are going to develop, which will include other sensors, like a GPS antenna, and which will be fully connected to the internet (via EDGE/UMTS), to local area networks (through the Ethernet port or through the Wifi module) and to the user's smartphone via Bluetooth. We plan to use such jackets, in conjunction with augmented-reality algorithms, to enhance historical city visits.

\textbf{Gesture-based interface}
This is basically a gesture-based controller for Power Point presentations or others desktop applications, that gives the user the ability to control a GUI using his gestures. Commands are passed through a simple socket, and then transformed in mouse or keyboards events. We exploit a client-server model, where our client is the Odroid-XU board, and the computer hosting the presentations acts as a server.

The board automatically connects to the remote server, via the following lines:
\begin{lstlisting}[frame=single]
int sockfd;
struct sockaddr_in serv_addr;
char buffer[1];
sockfd = socket(AF_INET, SOCK_STREAM, 0);
if (sockfd < 0) {
  cerr << "Error opening socket";
  exit(EXIT_FAILURE);
}
memset(&serv_addr, 0, sizeof(serv_addr));
serv_addr.sin_family = AF_INET;
serv_addr.sin_addr.s_addr = inet_addr(ip_address);
serv_addr.sin_port = htons(atoi(port_number));

if (connect(sockfd, (struct sockaddr*) &serv_addr, sizeof(serv_addr)) < 0) {
  cerr << "Unable to connect";
  exit(EXIT_FAILURE);
}
\end{lstlisting}
where \verb+ip_address+ and \verb+port_number+ are the server IP address port number. Once a gesture is recognized, the client sends the corresponding command on the socket, using the \verb+send+ primitive. In our implementation commands are coded with an unsigned char, thus allowing 255 different commands. Similarly, the server creates a socket and listens on \verb+port_number+. It then transforms the command into a sequence of mouse/keyboard events using \verb+xdotool+. For example, if a gesture corresponds to pressing the Enter key, the following command is executed:
\begin{lstlisting}[frame=single]
xdotool key KP_Enter
\end{lstlisting}
or, if the gesture corresponds to a mouse click, the server calls:
\begin{lstlisting}[frame=single]
xdotool click 1
\end{lstlisting}

Of course, more complex mouse/keyboard actions can be defined. In our demo (see Figure \ref{fig:projector}), we used a set of three gestures: the \textit{Point} gesture, to trigger an animation, and two \textit{Slide} gestures, to move backwards and forwards in a Power Point presentation.

\begin{figure}
\centering
\subfigure[The \textit{point} gesture reveals a description of the current artwork.]{
	\includegraphics[width=.65\columnwidth]{Figures/projector.jpg}
} \\
\subfigure[\textit{Slide} gestures let the user move forward and backwards.]{
	\includegraphics[width=.65\columnwidth]{Figures/projector2.jpg}
} \\
\caption{Gestures let the user control a virtual museum interface. A demo video is available at \url{http://www.lorenzobaraldi.com/files/EgoVision_HCI.wmv}}
\label{fig:projector}
\end{figure} 
%\input{Chapters/Chapter5} 
%\input{Chapters/Chapter6} 
%\input{Chapters/Chapter7} 

%----------------------------------------------------------------------------------------
%	THESIS CONTENT - APPENDICES
%----------------------------------------------------------------------------------------

\addtocontents{toc}{\vspace{2em}} % Add a gap in the Contents, for aesthetics

\appendix % Cue to tell LaTeX that the following 'chapters' are Appendices

% Include the appendices of the thesis as separate files from the Appendices folder
% Uncomment the lines as you write the Appendices

% Appendix A

\chapter{Publications} % Main appendix title

\label{AppendixA} % For referencing this appendix elsewhere, use \ref{AppendixA}

\lhead{Appendix A. \emph{Appendix Title Here}} % This is for the header on each page - perhaps a shortened title

The research activity conducted in the context of this thesis has led to some publications in international journals
and workshops. These are summarized below.

\begin{enumerate}
\item \textbf{L. Baraldi}, S. Alletto, G. Serra, R. Cucchiara. ``Interacting with Art: Ego-vision for Enriched Cultural Experience'', \textit{Machine Vision and Applications}, 2014. (Submitted)
\item \textbf{L. Baraldi}, F. Paci, G. Serra, L. Benini, R. Cucchiara. ``Gesture Recognition using Wearable Vision Sensors to Enhance Visitors’ Museum Experiences'', \textit{IEEE Sensors Journal}, 2014. (Submitted after minor revision)
\item \textbf{L. Baraldi}, F. Paci, G. Serra, L. Benini, R. Cucchiara. ``Gesture Recognition in Ego-Centric Videos using Dense Trajectories and Hand Segmentation'', in \textit{IEEE Computer Vision and Pattern Recognition (CVPR) Embedded Vision Workshop (EVW)}, 2014.
\item G. Serra, M. Camurri, \textbf{L. Baraldi}, M. Benedetti, R. Cucchiara. ``Hand Segmentation and Gesture Recognition in EGO-Vision'', in \textit{Proc. of ACM Multimedia International Workshop on Interactive Multimedia on Mobile and Portable Devices (IMMPD)}, 2013.
\end{enumerate}
%\input{Appendices/AppendixB}
%\input{Appendices/AppendixC}

\addtocontents{toc}{\vspace{2em}} % Add a gap in the Contents, for aesthetics

\backmatter

%----------------------------------------------------------------------------------------
%	BIBLIOGRAPHY
%----------------------------------------------------------------------------------------

\label{Bibliography}

\lhead{\emph{Bibliography}} % Change the page header to say "Bibliography"

\bibliographystyle{unsrtnat} % Use the "unsrtnat" BibTeX style for formatting the Bibliography

\bibliography{Bibliography} % The references (bibliography) information are stored in the file named "Bibliography.bib"

\end{document}  